%%%%%%%%%%%%%%%%%%%%%%%%%%%%%%%%%%%%%%%%%%%%%%%%%%%%%%%%%%%%%%%%%%%%%%%%
%%%%%%%%%%%%%%%%%%%%%% Simple LaTeX CV Template %%%%%%%%%%%%%%%%%%%%%%%%
%%%%%%%%%%%%%%%%%%%%%%%%%%%%%%%%%%%%%%%%%%%%%%%%%%%%%%%%%%%%%%%%%%%%%%%%

%%%%%%%%%%%%%%%%%%%%%%%%%%%%%%%%%%%%%%%%%%%%%%%%%%%%%%%%%%%%%%%%%%%%%%%%
%% NOTE: If you find that it says                                     %%
%%                                                                    %%
%%                           1 of ??                                  %%
%%                                                                    %%
%% at the bottom of your first page, this means that the AUX file     %%
%% was not available when you ran LaTeX on this source. Simply RERUN  %%
%% LaTeX to get the ``??'' replaced with the number of the last page  %%
%% of the document. The AUX file will be generated on the first run   %%
%% of LaTeX and used on the second run to fill in all of the          %%
%% references.                                                        %%
%%%%%%%%%%%%%%%%%%%%%%%%%%%%%%%%%%%%%%%%%%%%%%%%%%%%%%%%%%%%%%%%%%%%%%%%

%%%%%%%%%%%%%%%%%%%%%%%%%%%% Document Setup %%%%%%%%%%%%%%%%%%%%%%%%%%%%

% Don't like 10pt? Try 11pt or 12pt
\documentclass[10pt]{article}
\newcommand{\sapere}{\texttt{SAPERE}}
\newcommand{\doititle}[2]{\href{http://dx.doi.org/#1}{#2}}

% This is a helpful package that puts math inside length specifications
\usepackage{calc}
\usepackage[utf8]{inputenc} % Consente l'uso caratteri accentati italiani

% Simpler bibsection for CV sections
% (thanks to natbib for inspiration)
\makeatletter
\newlength{\bibhang}
\setlength{\bibhang}{1em}
\newlength{\bibsep}
 {\@listi \global\bibsep\itemsep \global\advance\bibsep by\parsep}
\newenvironment{bibsection}%
        {\vspace{-\baselineskip}\begin{list}{}{%
       \setlength{\leftmargin}{\bibhang}%
       \setlength{\itemindent}{-\leftmargin}%
       \setlength{\itemsep}{\bibsep}%
       \setlength{\parsep}{\z@}%
        \setlength{\partopsep}{0pt}%
        \setlength{\topsep}{0pt}}}
        {\end{list}\vspace{-.6\baselineskip}}
\makeatother

% Layout: Puts the section titles on left side of page
\reversemarginpar

%
%         PAPER SIZE, PAGE NUMBER, AND DOCUMENT LAYOUT NOTES:
%
% The next \usepackage line changes the layout for CV style section
% headings as marginal notes. It also sets up the paper size as either
% letter or A4. By default, letter was used. If A4 paper is desired,
% comment out the letterpaper lines and uncomment the a4paper lines.
%
% As you can see, the margin widths and section title widths can be
% easily adjusted.
%
% ALSO: Notice that the includefoot option can be commented OUT in order
% to put the PAGE NUMBER *IN* the bottom margin. This will make the
% effective text area larger.
%
% IF YOU WISH TO REMOVE THE ``of LASTPAGE'' next to each page number,
% see the note about the +LP and -LP lines below. Comment out the +LP
% and uncomment the -LP.
%
% IF YOU WISH TO REMOVE PAGE NUMBERS, be sure that the includefoot line
% is uncommented and ALSO uncomment the \pagestyle{empty} a few lines
% below.
%

%% Use these lines for letter-sized paper
\usepackage[paper=letterpaper,
            %includefoot, % Uncomment to put page number above margin
            marginparwidth=1.2in,     % Length of section titles
            marginparsep=.05in,       % Space between titles and text
            margin=1in,               % 1 inch margins
            includemp]{geometry}

%% Use these lines for A4-sized paper
%\usepackage[paper=a4paper,
%            %includefoot, % Uncomment to put page number above margin
%            marginparwidth=30.5mm,    % Length of section titles
%            marginparsep=1.5mm,       % Space between titles and text
%            margin=25mm,              % 25mm margins
%            includemp]{geometry}

%% More layout: Get rid of indenting throughout entire document
\setlength{\parindent}{0in}

%% This gives us fun enumeration environments. compactitem will be nice.
\usepackage{paralist}

%% Reference the last page in the page number
%
% NOTE: comment the +LP line and uncomment the -LP line to have page
%       numbers without the ``of ##'' last page reference)
%
% NOTE: uncomment the \pagestyle{empty} line to get rid of all page
%       numbers (make sure includefoot is commented out above)
%
\usepackage{fancyhdr,lastpage}
\pagestyle{fancy}
%\pagestyle{empty}      % Uncomment this to get rid of page numbers
\fancyhf{}\renewcommand{\headrulewidth}{0pt}
\fancyfootoffset{\marginparsep+\marginparwidth}
\newlength{\footpageshift}
\setlength{\footpageshift}
          {0.5\textwidth+0.5\marginparsep+0.5\marginparwidth-2in}
\lfoot{\hspace{\footpageshift}%
       \parbox{4in}{\, \hfill %
                    \arabic{page} of \protect\pageref*{LastPage} % +LP
%                    \arabic{page}                               % -LP
                    \hfill \,}}

% Finally, give us PDF bookmarks
\usepackage{color,hyperref}
\definecolor{darkblue}{rgb}{0.0,0.0,0.3}
\hypersetup{colorlinks,breaklinks,
            linkcolor=darkblue,urlcolor=darkblue,
            anchorcolor=darkblue,citecolor=darkblue}

%%%%%%%%%%%%%%%%%%%%%%%% End Document Setup %%%%%%%%%%%%%%%%%%%%%%%%%%%%


%%%%%%%%%%%%%%%%%%%%%%%%%%% Helper Commands %%%%%%%%%%%%%%%%%%%%%%%%%%%%

% The title (name) with a horizontal rule under it
% (optional argument typesets an object right-justified across from name
%  as well)
%
% Usage: \makeheading{name}
%        OR
%        \makeheading[right_object]{name}
%
% Place at top of document. It should be the first thing.
% If ``right_object'' is provided in the square-braced optional
% argument, it will be right justified on the same line as ``name'' at
% the top of the CV. For example:
%
%       \makeheading[\emph{Curriculum vitae}]{Your Name}
%
% will put an emphasized ``Curriculum vitae'' at the top of the document
% as a title. Likewise, a picture could be included:
%
%   \makeheading[\includegraphics[height=1.5in]{my_picutre}]{Your Name}
%
% the picture will be flush right across from the name.
\newcommand{\makeheading}[2][]%
        {\hspace*{-\marginparsep minus \marginparwidth}%
         \begin{minipage}[t]{\textwidth+\marginparwidth+\marginparsep}%
             {\large \bfseries #2 \hfill #1}\\[-0.15\baselineskip]%
                 \rule{\columnwidth}{1pt}%
         \end{minipage}}

% The section headings
%
% Usage: \section{section name}
%
% Follow this section IMMEDIATELY with the first line of the section
% text. Do not put whitespace in between. That is, do this:
%
%       \section{My Information}
%       Here is my information.
%
% and NOT this:
%
%       \section{My Information}
%
%       Here is my information.
%
% Otherwise the top of the section header will not line up with the top
% of the section. Of course, using a single comment character (%) on
% empty lines allows for the function of the first example with the
% readability of the second example.
\renewcommand{\section}[2]%
        {\pagebreak[3]\vspace{1.3\baselineskip}%
         \phantomsection\addcontentsline{toc}{section}{#1}%
         \hspace{0in}%
         \marginpar{
         \raggedright \scshape #1}#2}

% An itemize-style list with lots of space between items
\newenvironment{outerlist}[1][\enskip\textbullet]%
        {\begin{itemize}[#1]}{\end{itemize}%
         \vspace{-.6\baselineskip}}

% An environment IDENTICAL to outerlist that has better pre-list spacing
% when used as the first thing in a \section
\newenvironment{lonelist}[1][\enskip\textbullet]%
        {\vspace{-\baselineskip}\begin{list}{#1}{%
        \setlength{\partopsep}{0pt}%
        \setlength{\topsep}{0pt}}}
        {\end{list}\vspace{-.6\baselineskip}}

% An itemize-style list with little space between items
\newenvironment{innerlist}[1][\enskip\textbullet]%
        {\begin{compactitem}[#1]}{\end{compactitem}}

% An environment IDENTICAL to innerlist that has better pre-list spacing
% when used as the first thing in a \section
\newenvironment{loneinnerlist}[1][\enskip\textbullet]%
        {\vspace{-\baselineskip}\begin{compactitem}[#1]}
        {\end{compactitem}\vspace{-.6\baselineskip}}

% To add some paragraph space between lines.
% This also tells LaTeX to preferably break a page on one of these gaps
% if there is a needed pagebreak nearby.
\newcommand{\blankline}{\quad\pagebreak[3]}
\newcommand{\halfblankline}{\quad\vspace{-0.5\baselineskip}\pagebreak[3]}

% Uses hyperref to link DOI
\newcommand\doilink[1]{\href{http://dx.doi.org/#1}{#1}}
\newcommand\doi[1]{doi:\doilink{#1}}

% For \url{SOME_URL}, links SOME_URL to the url SOME_URL
\providecommand*\url[1]{\href{#1}{#1}}
% Same as above, but pretty-prints SOME_URL in teletype fixed-width font
\renewcommand*\url[1]{\href{#1}{\texttt{#1}}}

% For \email{ADDRESS}, links ADDRESS to the url mailto:ADDRESS
\providecommand*\email[1]{\href{mailto:#1}{#1}}
% Same as above, but pretty-prints ADDRESS in teletype fixed-width font
%\renewcommand*\email[1]{\href{mailto:#1}{\texttt{#1}}}

%\providecommand\BibTeX{{\rm B\kern-.05em{\sc i\kern-.025em b}\kern-.08em
%    T\kern-.1667em\lower.7ex\hbox{E}\kern-.125emX}}
%\providecommand\BibTeX{{\rm B\kern-.05em{\sc i\kern-.025em b}\kern-.08em
%    \TeX}}
\providecommand\BibTeX{{B\kern-.05em{\sc i\kern-.025em b}\kern-.08em
    \TeX}}
\providecommand\Matlab{\textsc{Matlab}}

%%%%%%%%%%%%%%%%%%%%%%%% End Helper Commands %%%%%%%%%%%%%%%%%%%%%%%%%%%

%%%%%%%%%%%%%%%%%%%%%%%%% Begin CV Document %%%%%%%%%%%%%%%%%%%%%%%%%%%%

\begin{document}
\makeheading{Danilo Pianini}

\section{Contacts}
%
% NOTE: Mind where the & separators and \\ breaks are in the following
%       table.
%
% ALSO: \rcollength is the width of the right column of the table
%       (adjust it to your liking; default is 1.85in).
%
\newlength{\rcollength}\setlength{\rcollength}{2.5in}%
%
\begin{tabular}[t]{@{}p{\textwidth-\rcollength}p{\rcollength}}
%\href{http://www.cse.osu.edu/}%
%     {Department of Computer Science and Engineering} & \\
%\href{http://www.osu.edu/}{The Ohio State University}
                           
Via Venezia, 52          & \textit{Mobile:} +39 320 4136573\\
47521 Cesena (FC)          & \textit{E-mail:} \email{danilo.pianini@unibo.it}\\
Italy                      & \textit{WWW:} \href{http://www.danilopianini.org/}{www.danilopianini.org}\\
\end{tabular}

\section{Objective}
%
Placement in an academic position that allows for advanced research in pervasive computing (i.e., modeling, analysis, design, and verification) with a particular focus on simulation.

\section{Citizenship}
Italy

\section{Research Themes}
%
My current research is focussed on the engineering aspects of pervasive computing, with the goal of providing a robust, easy, and coherent chain of tools and procedures that can lead to robust, adaptive, self-healing, and possibly evolving software ecosystems. I recently got fascinated by aggregate programming, namely all the languages and techniques that enable the programming of groups of devices as a single entity distributed in space and time.

\section{Education}
%
\href{http://www.deis.unibo.it/DEIS/default.htm}{\textbf{Dipartimento di Ingegneria Elettronica, Informatica e delle Telecomunicazioni, Università di Bologna}}, Bologna (BO), Italy
\begin{outerlist}
\item[] Ph.D. in
        \href{http://www.cse.unibo.it/en/phd-program}
             {Electronics, Computer Science and Telecommunications Engineering},
%              started in January 2012, covered by Italian ministerial grant.
        \begin{innerlist}
        \item Thesis Title: \href{http://amsdottorato.unibo.it/7000/}{\emph{Engineering Complex Computational Ecosystems}}
%         \item Thesis Proposal: \emph{Cooperative Task Processing}
%         \item Candidacy: \emph{Research
%             Problems in Distributed Control for Energy Systems}
        \item Supervisor:
              \href{http://mirkoviroli.apice.unibo.it/}{Prof. Mirko Viroli}
        \item Tutor:
              \href{http://lia.deis.unibo.it/Staff/AntonioNatali/}{Prof. Antonio Natali}
        \item External reviewer:
              \href{http://cui.unige.ch/~dimarzo/}{Prof. Giovanna di Marzo Serugendo}
        \item External reviewer:
              \href{http://www.simondobson.org/}{Prof. Simon Dobson}
        \item Area of Study: Pervasive computing
        \end{innerlist}

\halfblankline{}
\end{outerlist}
\href{http://www.ing2.unibo.it/Ingegneria+Cesena/default.htm}{\textbf{Seconda Facoltà di Ingegneria, Università di Bologna}}, Cesena (FC), Italy
\begin{outerlist}
\item[] M.S.,
        \href{http://www.ing2.unibo.it/Ingegneria+Cesena/default.htm}
             {Computer Engineering}, March 2011
        \begin{innerlist}
        \item \emph{110L/110 - Magna cum Laude}
        \item Thesis Topic: \emph{A Framework for Simulation of Pervasive Services Ecosystems}
        \item Supervisor:
              \href{http://mirkoviroli.apice.unibo.it/}
                   {Dr. Mirko Viroli}
        \item Area of Study: Computational Models
        \end{innerlist}
        
\item[] B.S.,
        \href{http://www.ing2.unibo.it/Ingegneria+Cesena/default.htm}
             {Computer Engineering}, October 2008
        \begin{innerlist}
        \item Thesis Topic: \emph{From Swarm Intelligence to Self-Organising Coordination: a Pervasive Scenarios Application}
        \item Supervisor:
              \href{http://apice.unibo.it/xwiki/bin/view/AndreaOmicini/}
                   {Prof. Andrea Omicini}
        \item Area of Study: Distributed Systems
        \end{innerlist}
    
\halfblankline
\end{outerlist}
\href{http://www.iiseinaudi.it/}{\textbf{ITCG L. Einaudi}}, Novafeltria (RN), Italy
\begin{outerlist}
 \item [ ] Scientific high school, focus on biology, July 2005
        \begin{innerlist}
	  \item 100/100
        \end{innerlist}
\end{outerlist}

\vspace{0.1in}
\section{Publications ordered by time}
\renewcommand{\section}[2]{}
\nocite{*}
\vspace{-0.29in}
\bibliographystyle{IEEEtran}
\bibliography{bibliography}
\vspace{0.1in}

\renewcommand{\section}[2]%
        {\pagebreak[3]\vspace{1.3\baselineskip}%
         \phantomsection\addcontentsline{toc}{section}{#1}%
         \hspace{0in}%
         \marginpar{
         \raggedright \scshape #1}#2}

\section{Certifications}
\href{http://www.miur.gov.it/abilitazione-scientifica-nazionale}{Abilitazione Scientifica Nazionale al ruolo di professore di II fascia}\\
Italian Ministry of Education, Universities and Research\\
\href{https://asn16.cineca.it/pubblico/miur/esito/09\%252FH1/2/5/84066/giudizi}{Starting 2018-07-26, ending 2024-07-26}

\section{Editorial activity}
\href{https://www.hindawi.com/journals/sp/}{\textbf{Scientific programming}}: member of the editorial board --- since 2017

\section{{\color{black}Service in international conferecences}}
% !TEX root = curriculum.tex
\section{{\color{black}Chairing in international conferences}}
\halfblankline \\
\href{https://conf.researchr.org/home/acsos-2022/}{3rd International Conference on Autonomic Computing and Self-Organizing Systems
(ACSOS 2022)}
\\ Special Event Chair \\
\halfblankline \\
\href{https://conf.researchr.org/home/acsos-2021}{2nd IEEE International Conference on Autonomic Computing and Self-Organizing Systems 
(ACSOS 2021)}
\\ Program Committee chair \\
\halfblankline \\
\href{https://ngps2019.github.io/}{Next Generation Programming Languages and Systems (NGPS 2019) --- 
Track of the 34th ACM Symposium on Applied Computing (SAC 2019)}
\\ Track chair \\
\halfblankline \\
\href{https://saso2018.fbk.eu/}{12th IEEE International Conference on Self-Adaptive and Self-Organizing (SASO 2018)}
\\ Workshops and tutorials chair \\
\halfblankline \\
\href{http://icac2018.informatik.uni-wuerzburg.de/committees/organization-committee/}{15th IEEE International Conference on Autonomic Computing (ICAC 2018)}
\\ Workshops and tutorials chair \\
\halfblankline \\
\href{http://sac-cas2018.apice.unibo.it/referees.html}{Collective and Cooperative Systems --- Special Track of the 33rd ACM Symposium on Applied Computing (SAC 2018)}
\\ Session chair \\
\halfblankline \\
\href{http://apice.unibo.it/xwiki/bin/view/ALP4IoT2016/WebHome}{1st workshop on Architectures, Languages and Paradigms for IoT (ALP4IoT 2017)}
\\ Program Committee chair \\

\section{{\color{black}Participation in panels}}
\halfblankline \\
\href{https://2022.acsos.org/details/acsos-2022-papers/36/Hot-Topics-and-Current-Trends-in-ACSOS-Research}{3rd International Conference on Autonomic Computing and Self-Organizing Systems
    (ACSOS 2022)}
\\ \href{https://danysk.github.io/Slides-2022-ACSOS-Panel/#/}{Hot Topics and Current Trends in ACSOS Research} \\

\section{{\color{black}Participation in program committees of international conferences}}
\halfblankline \\
\href{https://icaart.scitevents.org/}{15th International Conference on Agents and Artificial Intelligence
(ICAART 2023)}
\\ Program Committee member \\
\halfblankline \\
\href{https://sissy.telecom-paristech.fr/}{Workshop on Self-Improving Systems Integration (SISSY 2022)}
\\ Program Committee member \\
\halfblankline \\
\href{https://conf.researchr.org/home/acsos-2022/}{3rd International Conference on Autonomic Computing and Self-Organizing Systems
(ACSOS 2022)}
\\ Program Committee member \\
\halfblankline \\
\href{https://ifm22.si.usi.ch/}{17th International Conference on integrated Formal Methods
(iFM 2022)}
\\ Artifact Evaluation Committee member \\
\halfblankline \\
\href{https://discoli-workshop.github.io/2022/}{1st Workshop on DIStributed COLlective Intelligence
(DISCOLI 2022)}
\\ Program Committee member \\
\halfblankline \\
\href{https://www.discotec.org/2022/coordination}{1st Workshop on Adaptive, Learning PervAsive Computing Applications
(ALPACA 2022)}
\\ Program Committee member \\
\halfblankline \\
\href{https://www.discotec.org/2022/coordination}{24th International Conference on Coordination Models and Languages 
(COORDINATION 2022)}
\\ Artefact Evaluation Committee member \\
\halfblankline \\
\href{http://www.icaart.org/?y=2022}{14th International Conference on Agents and Artificial Intelligence 
(ICAART 2022)}
\\ Program Committee member \\
\halfblankline \\
\href{http://www.icaart.org/?y=2021}{13th International Conference on Agents and Artificial Intelligence 
(ICAART 2021)}
\\ Program Committee member \\
\halfblankline \\
\href{http://archive.vn/wip/38Ah6}{5th Workshop on Engineering Collective Adaptive Systems (eCAS 2020)}
\\ Program Committee member \\
\halfblankline \\
\href{https://conf.researchr.org/home/acsos-2020}{1st IEEE International Conference on Autonomic Computing and Self-Organizing Systems (ACSOS 2020)}
\\ Program Committee member \\
\halfblankline \\
\href{https://https://ijcai20.org/}{29th International Joint Conference on Artificial Intelligence (IJCAI 2020)}
\\ Program Committee member \\
\halfblankline \\
\href{https://aamas2020.conference.auckland.ac.nz/program-committee-members/}{International Conference on Autonomous Agents and Multi-Agent Systems 2020 (AAMAS 2020)}
\\ Program Committee member \\
\halfblankline \\
\href{http://www.discotec.org/2020/coordination}{COORDINATION 2020 - 22nd International Conference on Coordination Models and Languages}
\\ Program Committee member \\
\halfblankline \\
\href{http://www.icaart.org/?y=2020}{12th International Conference on Agents and Artificial Intelligence 
(ICAART 2020)}
\\ Program Committee member \\
\halfblankline \\
\href{http://apice.unibo.it/xwiki/bin/view/ECAS2019/Committees}{4th Workshop on Engineering Collective Adaptive Systems, (eCAS 2019)}
\\ Program Committee member \\
\halfblankline \\
\href{http://aamas2019.encs.concordia.ca/}{International Conference on Autonomous Agents and 
Multiagent Systems (AAMAS 2019)}
\\ Program Committee member \\
\halfblankline \\
\href{http://www.icaart.org/?y=2019}{11th International Conference on Agents and Artificial Intelligence 
(ICAART 2019)}
\\ Program Committee member \\
\halfblankline \\
\href{http://www.discotec.org/2019/coordination}{COORDINATION 2019 - 21st International Conference on Coordination Models and Languages}
\\ Program Committee member \\
\halfblankline \\
\href{http://diid.unipa.it/roboticslab/woa2018/}{XIX Workshop "From Objects to Agents" (WOA 2018)}
\\ Program Committee member \\
\halfblankline \\
\href{http://sac-cas2018.apice.unibo.it/referees.html}{Collective and Cooperative Systems --- Special Track of the 33rd ACM Symposium on Applied Computing (SAC 2018)}
\\ Program Committee member \\
\halfblankline \\
\href{http://apice.unibo.it/xwiki/bin/view/ECAS2017/WebHome}{2nd eCAS Workshop on Engineering Collective Adaptive Systems (eCAS 2017)}
\\ Program Committee member \\
\halfblankline \\
\href{http://woa2017.unirc.it/}{XVIII WORKSHOP "From Objects to Agents" (WOA 2017)}
\\ Program Committee member \\


\section{Reviewing for international journals}
% ! TeX root = curriculum.tex
\href{https://www.sciencedirect.com/journal/science-of-computer-programming}{\textbf{Science of Computer Programming} (ISSN: 01676423)}, 2022
\\ \halfblankline \\
\href{https://www.computer.org/csdl/journal/sc}{\textbf{IEEE Transactions on Services Computing} (ISSN: 19391374)}, 2022
\\ \halfblankline \\
\href{https://www.iospress.com/catalog/journals/intelligenza-artificiale}{\textbf{IOS Press Intelligenza Artificiale} (ISSN: 17248035, 22110097)}, 2022
\\ \halfblankline \\
\href{https://www.journals.elsevier.com/expert-systems-with-applications}{\textbf{Elsevier Expert Systems With Applications} (ISSN: 09574174)}, 2021
\\ \halfblankline \\
\href{https://www.mdpi.com/journal/network}{\textbf{MDPI Network} (ISSN: 2673-8732)}, 2021
\\ \halfblankline \\
\href{https://www.journals.elsevier.com/journal-of-information-security-and-applications}{\textbf{Elsevier Journal of Information Security and Applications} (ISSN: 22142126, 22142134)}, 2021
\\ \halfblankline \\
\href{https://www.mdpi.com/journal/electronics}{\textbf{MDPI Electronics} (ISSN: 20799292)}, 2020-2021
\\ \halfblankline \\
\href{https://www.sciencedirect.com/journal/future-generation-computer-systems}{\textbf{Elsevier Future Generation Computer Systems} (ISSN: 0167739X)}, 2020--2023
\\ \halfblankline \\
\href{https://www.journals.elsevier.com/pervasive-and-mobile-computing}{\textbf{Elsevier Pervasive and Mobile Computing} (ISSN: 15741192)}, 2020
\\ \halfblankline \\
\href{http://www.mdpi.com/journal/sensors}{\textbf{MDPI Sensors} (ISSN: 14243210, 14248220)}, 2016-2020
\\ \halfblankline \\
\href{https://link.springer.com/journal/10462}{\textbf{Springer Artificial Intelligence Review (AIRE)} (ISSN: 02692821, 15737462)}, 2019
\\ \halfblankline \\
\href{https://www.hindawi.com/journals/mpe/}{\textbf{Hindawi Mathematical Problems in Engineering} (ISSN: 1024123X, 15635147)}, 2017--2019
\\ \halfblankline \\
\href{http://www.mdpi.com/journal/applsci}{\textbf{MDPI Applied sciences} (ISSN: 20763417)}, 2018
\\ \halfblankline \\
\href{https://academic.oup.com/comjnl}{\textbf{Oxford University Press The Computer Journal} (ISSN: 00104620, 14602067)}, 2018
\\ \halfblankline \\
\href{https://www.journals.elsevier.com/artificial-intelligence-in-medicine/}{\textbf{Elsevier Artificial Intelligence in Medicine} (ISSN: 09333657, 18732860)}, 2018
\\ \halfblankline \\
\href{https://www.journals.elsevier.com/computational-and-structural-biotechnology-journal/}{\textbf{Elsevier Computational and Structural Biotechnology Journal} (ISSN: 20010370)}, 2016
\\ \halfblankline \\
\href{http://cacm.acm.org/}{\textbf{Communications of the ACM} (ISSN: 00010782, 15577317)}, 2016


\section{Talks in international conferecences}
% ! TeX root = curriculum.tex
\href{https://danysk.github.io/slides-2023-asmecc/}{Infrastructures for the Edge-Cloud Continuum on a Small Scale: a Practical Case Study} \\
\href{https://asmecc-workshop.github.io/2023/}{\textit{1st ASMECC Workshop on Autonomic and Self-* Management for the Edge-Cloud Continuum (ASMECC 2023)}}
\\ \halfblankline \\
\href{https://danysk.github.io/slides-2023-dais-loadshift/}{Runtime Load-Shifting of Distributed Controllers Across Networked Devices} \\
\href{https://www.discotec.org/2023/dais.html}{\textit{23rd International Conference on Distributed Applications and Interoperable Systems (DAIS 2023)}}
\\ \halfblankline \\
\href{https://danysk.github.io/Slides-2022-ACSOS-BoundedElection/}{Self-stabilising Priority-Based Multi-Leader Election and Network Partitioning} \\
\href{https://2022.acsos.org/}{\textit{3rd IEEE International Conference on Autonomic Computing and Self-Organizing Systems - ACSOS 2022}}
\\ \halfblankline \\
\href{https://danysk.github.io/Slides-2022-Coordination-SpaceFluid/}{Space-Fluid Adaptive Sampling: a Field-Based, Self-Organising Approach} \\
\href{https://danysk.github.io/Slides-2022-Coordination-SpaceFluid/}{\textit{24st International Conference on Coordination Models and Languages (COORDINATION 2022)}}
\\ \halfblankline \\
\href{https://alchemistsimulator.github.io/tutorials/basics/index.html}{Simulation of Large Scale Computational Ecosystems with Alchemist: A Tutorial} \\
\href{https://www.discotec.org/2023/dais.html}{\textit{21st International Conference on Distributed Applications and Interoperable Systems (DAIS 2021)}}
\\ \halfblankline \\
\href{https://danysk.github.io/Slides-2020-Coordination-TimeFluid/}{Time-Fluid Field-Based Coordination} \\
\href{http://www.discotec.org/2020/coordination.html}{\textit{22nd International Conference on Coordination Models and Languages (COORDINATION 2020)}}
\\ \halfblankline \\
\href{https://danysk.github.io/Slides-2019-Coordination-SCR/}{Self-organising Coordination Regions: a pattern for edge computing} \\
\href{http://www.discotec.org/2019/coordination.html}{\textit{21st International Conference on Coordination Models and Languages (COORDINATION 2019)}}
\\ \halfblankline \\
\href{https://danysk.github.io/Slides-2019-eCAS-security/}{Security in Collective Adaptive Systems: a Roadmap} \\
\href{https://apice.unibo.it/xwiki/bin/view/ECAS2019/}{\textit{4th eCAS Workshop on Engineering Collective Adaptive Systems (eCAS 2019)}}
\\ \halfblankline \\
\href{https://danysk.github.io/Slides-2018-BISS/}{Computing at the Aggregate Level} \\
\href{https://www.biss-institute.com/wp-content/uploads/2018/07/For-more-information-download-the-brochure.pdf}{\textit{Workshop ``Making the smart city safe for citizens:
The case of smart energy and mobility''}}
\\ \halfblankline \\
\href{https://www.slideshare.net/DanySK/engineering-the-aggregate-talk-at-software-engineering-for-intelligent-and-autonomous-systems-sefias-dagstuhl-2018}{Engineering the Aggregate} \\
\href{https://www.hpi.uni-potsdam.de/giese/public/selfadapt/dagstuhl-seminars/sefias/}{\textit{GI Dagstuhl Seminar ``Software Engineering for Intelligent and Autonomous Systems'' (SEfIAS 2018)}}
\\ \halfblankline \\
Themes and Challenges in Engineering CAS \\
Panelist at the \href{http://apice.unibo.it/xwiki/bin/view/ECAS2017/WebHome}{\textit{2nd eCAS Workshop on Engineering Collective Adaptive Systems (eCAS 2017)}}
\\ \halfblankline \\
\href{https://www.slideshare.net/DanySK/practical-aggregate-programming-with-protelis-saso2017}{Practical Aggregate Programming with Protelis} \\
Tutorial at the \href{http://apice.unibo.it/xwiki/bin/view/ECAS2017/WebHome}{\textit{11th IEEE International Conference on Self-Adaptive and Self-Organizing Systems (SASO 2017)}}
\\ \halfblankline \\
\href{https://www.slideshare.net/DanySK/2nd-ecas-workshop-on-engineering-collective-adaptive-systems}{Towards a Foundational API for Resilient Distributed Systems Design} \\
\href{http://apice.unibo.it/xwiki/bin/view/ECAS2017/WebHome}{\textit{2nd eCAS Workshop on Engineering Collective Adaptive Systems (eCAS 2017)}}
\\ \halfblankline \\
\href{https://www.slideshare.net/DanySK/simulating-largescale-aggregate-mass-with-alchemist-and-scala}{Simulating Large-scale Aggregate MASs with Alchemist and Scala} \\
\href{https://fedcsis.org/2016/mass}{\textit{10th International Workshop on Multi-Agent Systems and Simulation (MAS\&S 2016)}}
\\ \halfblankline \\
\href{https://www.slideshare.net/DanySK/computational-fields-meet-augmented-reality-perspectives-and-challenges}{Computational Fields meet Augmented Reality: Perspectives and Challenges} \\
\href{http://www.spatial-computing.org/scopes}{\textit{1st Workshop on Spatial and COllective PErvasive Computing Systems (SCOPES 2015)}}
\\ \halfblankline \\
\href{http://apice.unibo.it/xwiki/bin/download/PRIMA2015/Demos/Pianini.pdf}{Engineering multi-agent systems with aggregate computing} \\
Demo at the \href{http://apice.unibo.it/xwiki/bin/view/PRIMA2015/Demos}{\textit{18th Conference on Principles and Practice of Multi-Agent Systems (PRIMA 2015)}}
\\ \halfblankline \\
\href{https://www.slideshare.net/DanySK/extending-the-gillespies-stochastic-simulation-algorithm-for-integrating-discreteevent-and-multiagent-based-simulation}{Extending the Gillespie's Stochastic Simulation Algorithm for Integrating Discrete-Event and Multi-Agent Based Simulation} \\
\href{http://www.springer.com/gp/book/9783319314464}{\textit{XVI International Workshop on Multi-Agent Based Simulation (MABS 2015)}}
\\ \halfblankline \\
\href{https://www.slideshare.net/DanySK/sac-47194849}{Protelis: Practical Aggregate Programming} \\
\href{https://www.sigapp.org/sac/sac2015/}{\textit{The 30th ACM/SIGAPP Symposium On Applied Computing (SAC 2015)}}
\\ \halfblankline \\
\href{https://www.slideshare.net/DanySK/gradientbased-selforganisation-patterns-of-anticipative-adaptation}{Gradient-based Self-organisation Patterns of Anticipative Adaptation} \\
\href{http://saso2012.univ-lyon1.fr/}{\textit{6th IEEE International Conference on Self-Adaptive and Self-Organizing Systems (SASO 2012)}}
\\ \halfblankline \\
\href{http://apice.unibo.it/xwiki/bin/view/Talks/PianiniMass2011}{A Chemical Inspired Simulation Framework for Pervasive Services Ecosystems} \\
\href{https://fedcsis.org/2011/}{\textit{5th International Workshop on Multi-Agent Systems and Simulation (MAS\&S 2011)}}
\\ \halfblankline \\
\href{http://apice.unibo.it/xwiki/bin/view/Talks/PianiniVirrusoSASO10}{Self Organization in Coordination Systems using a WordNet-based Ontology} \\
\href{http://www.inf.u-szeged.hu/projectdirs/saso10/}{\textit{Fourth IEEE International Conference on Self-Adaptive and Self-Organizing Systems (SASO 2010)}}


\section{Other talks}
\href{https://www.slideshare.net/DanySK/continuous-integration-and-delivery-77394349}{Continuous integration and delivery} \\
Seminar for the ``\href{http://apice.unibo.it/xwiki/bin/view/Courses/PPS1617}{Programming and development paradigms}'' course, 2016
\\ \halfblankline \\
\href{https://www.slideshare.net/DanySK/democratic-process-and-electronic-platforms-concerns-of-an-engineer}{Democratic process and electronic platforms: concerns of an engineer} \\
Workshop ``\href{https://sites.google.com/site/futureofdemocracy2016/home/workshop}{The Future of Democracy}'', 2016
\\ \halfblankline \\
\href{https://www.slideshare.net/DanySK/software-development-made-serious}{Software development made serious} \\
Seminar for the ``\href{http://apice.unibo.it/xwiki/bin/view/Courses/ISAC1516}{Adaptive complex software systems engineering}'' course, 2016
\\ \halfblankline \\
\href{https://www.slideshare.net/DanySK/engineering-complex-computational-ecosystems-phd-defense}{Engineering Complex Computational Ecosystems} \\
PhD defense, 2015
\\ \halfblankline \\
\href{https://www.slideshare.net/DanySK/engineering-computational-ecosystems-2nd-year-ph-d-seminar}{Engineering computational ecosystems} \\
2nd year PhD seminar, 2013
\\ \halfblankline \\
\href{http://campus.unibo.it/115195/}{From Engineer to Alchemist, There and Back Again: An Alchemist Tale} \\
Seminar for the ``\href{http://apice.unibo.it/xwiki/bin/view/Courses/LsaLm1213}{Laboratory of systems and applications LM}'' course, 2012
\\ \halfblankline \\
\href{https://www.slideshare.net/DanySK/engineering-computational-ecosystems}{Engineering computational ecosystems} \\
Vieni via con noi, 2012, Cesena
\\ \halfblankline \\
\href{https://www.slideshare.net/DanySK/recipes-for-sabayon-cook-your-own-linux-distro-within-two-hours}{Recipes for Sabayon: cook your own Linux distro within two hours} \\
Linux Day 2012, Cesena
\\ \halfblankline \\
\href{http://campus.unibo.it/83921/}{The simulation alchemy} \\
Seminar for the ``\href{http://apice.unibo.it/xwiki/bin/view/Courses/LsaLm1112}{Laboratory of systems and applications LM}'' course, 2011
\\ \halfblankline \\
\href{http://apice.unibo.it/xwiki/bin/view/Talks/PianiniWoa2011}{A Simulation Framework for Pervasive Service Ecosystems} \\
\href{http://www.inf.u-szeged.hu/projectdirs/saso10/}{\textit{XII Workshop ``Dagli Oggetti agli Agenti''} (WOA 2010)}


\section{Teaching}
\href{http://www.unibo.it/Portale/default.htm}{\textbf{Alma Mater Studiorum Università di Bologna}}, Bologna (BO), Italy
\begin{outerlist}
\item[] \textit{Post-doc} \hfill \textbf{since January 2015}
    \begin{innerlist}
      \item Co-Supervisor \href{http://amslaurea.unibo.it/16841/}{Nicolas Barilari's bachelor 
thesis: \textit{Programmazione Reattiva in Kotlin su sistemi Android}}, 2018.
      \item Co-Supervisor \href{http://amslaurea.unibo.it/16788/}{Luca Casamenti's bachelor 
thesis: \textit{Il linguaggio Ceylon}}, 2018.
      \item Co-Supervisor \href{http://amslaurea.unibo.it/15730/}{Davide Bondi's bachelor 
thesis: \textit{Protocollo LoRaWAN e IoT: interfacciamento con Java e sperimentazione su 
comunicazioni indoor}}, 2018.
      \item Professor for
\href{
https://bitbucket.org/danysk/courses-2018-developing-maintaining-and-sharing-software-tools/download
s/}{\textit{Developing, Maintaining, and Sharing Software Tools for Research}}, doctoral course for 
the \href{https://www.unibo.it/en/teaching/phd/2017-2018/data-science-and-computation}{PhD in Data 
Science and Computation}, XXXIII cycle, 2018.
      \item Co-Supervisor \href{http://amslaurea.unibo.it/?????/}{Matteo Magnani's bachelor 
thesis: \textit{Design e implementazione di un sistema di grid computing per il simulatore 
Alchemist}}, 2017.
      \item Co-Supervisor \href{http://amslaurea.unibo.it/14682/}{Niccolò Maltoni's bachelor thesis: 
\textit{Progettazione object-oriented di un'interfaccia grafica JavaFX per il simulatore 
Alchemist}}, 2017.
      \item Co-Supervisor \href{http://amslaurea.unibo.it/14673/}{Luca Semprini's bachelor thesis: \textit{Una panoramica su Kotlin: il nuovo linguaggio per lo sviluppo di applicazioni Android}}, 2017.
      \item Co-Supervisor \href{http://amslaurea.unibo.it/14329/}{Andrea Placuzzi's bachelor thesis: \textit{Integrazione dei formati di navigazione GPS standard in Alchemist}}, 2017.
      \item Co-Supervisor \href{http://amslaurea.unibo.it/14019/}{Giacomo Scaparrotti's bachelor thesis: \textit{Studio delle prestazioni del simulatore Alchemist: ottimizzazione di routing e caching}}, 2017.
      \item Co-Supervisor \href{http://amslaurea.unibo.it/13090/}{Matteo Francia's master thesis: \textit{A Foundational Library for Aggregate Programming}}, 2017.
      \item Contract Professor for the course ``\href{http://apice.unibo.it/xwiki/bin/view/Courses/OOP1718}{Object-Oriented Programming}'', 2017.
      \item Contract Professor for the course ``\href{http://apice.unibo.it/xwiki/bin/view/Courses/OOP1617}{Object-Oriented Programming}'', 2016.
      \item Co-Supervisor \href{http://amslaurea.unibo.it/12310/}{Elisa Casadio's bachelor thesis: \textit{Revisione e refactoring dell'interfaccia utente del simulatore Alchemist}}, 2016.
      \item Co-Supervisor \href{http://amslaurea.unibo.it/12503/}{Gianluca Grossi's bachelor thesis: \textit{Sviluppo di plugin per IntelliJ IDEA}}, 2016.
      \item Co-Supervisor \href{http://amslaurea.unibo.it/10519/}{Simone Costanzi's master thesis: \textit{Integrazione di piattaforme d'esecuzione e simulazione in una toolchain Scala per aggregate programming}}, 2016.
      \item Co-Supervisor \href{http://amslaurea.unibo.it/10481/}{Giovanni Romio's bachelor thesis: \textit{Backport di una applicazione da Java 8 a Java 7}}, 2016.
      \item Contract Professor for the course ``\href{http://apice.unibo.it/xwiki/bin/view/Courses/ISAC1516}{Complex Adaptive Software System Engineering}'', 2016.
      \item Seminar ``\href{https://www.slideshare.net/DanySK/software-development-made-serious}{Software development made serious}'', 2016
      \item Teaching assistant for the course ``\href{http://apice.unibo.it/xwiki/bin/view/Courses/OOP1516}{Object-Oriented Programming}'', 2015.
      \item Contract Professor for the course ``\href{http://www.apice.unibo.it/xwiki/bin/view/Courses/FINFA1415/}{Computer Science Foundations A}'', 2015.
      \item Teaching assistant for the course ``\href{http://apice.unibo.it/xwiki/bin/view/Courses/ISAC1415}{Complex Adaptive Software System Engineering}'', 2015.
    \end{innerlist}
\item[] \textit{PhD Student} \hfill \textbf{January 2012 to December 2014}
    \begin{innerlist}
      \item Teaching assistant for the course ``\href{http://apice.unibo.it/xwiki/bin/view/Courses/OOP1415}{Object Oriented Programming}'', 2014.
      \item Teaching assistant for the course ``\href{http://apice.unibo.it/xwiki/bin/view/Courses/ISAC1314}{Complex Adaptive Software System Engineering }'', 2014.
      \item Teaching assistant for the course ``\href{http://apice.unibo.it/xwiki/bin/view/Courses/OOP1314}{Object Oriented Programming}'', 2013.
      \item Teaching assistant for the course ``\href{http://www.apice.unibo.it/xwiki/bin/view/Courses/FINFA1213/}{Computer Science Foundations A}'', 2013.
      \item Co-Supervisor \href{http://amslaurea.unibo.it/7990/}{Davide Ensini's master thesis: \textit{Spatial computing per smart devices}}, 2014.
      \item Co-Supervisor \href{http://amslaurea.unibo.it/6927/}{Luca Nenni's master thesis: \textit{Simulazioni realistiche di algoritmi di Crowd Steering}}, 2014.
      \item Co-Supervisor \href{http://amslaurea.unibo.it/5293/}{Enrico Polverelli's master thesis: \textit{Simulazione di algoritmi di auto-organizzazione basati su gradiente computazionale in Alchemist}}, 2012.
      \item Co-Supervisor \href{http://amslaurea.unibo.it/4217/}{Andrea Dallatana's master thesis: \textit{BDI agents for Real Time Strategy games}}, 2012.
      \item Co-Supervisor \href{http://amslaurea.unibo.it/4088/}{Francesca Cioffi's master thesis: \textit{Algoritmi gradient-based per la modellazione e simulazione di sistemi auto-organizzanti}}, 2012.
      \item Co-Supervisor \href{http://amslaurea.unibo.it/4074/}{Paolo Contessi's master thesis: \textit{Supporting semantic web technologies in the pervasive service ecosystems middleware}}, 2012.
      \item Co-Supervisor \href{http://www.alice.unibo.it/xwiki/bin/view/Theses/ProntiAlchemistSapere/}{Giacomo Pronti's master thesis: \textit{Simulazione di ecosistemi di servizi pervasivi con supporto ad annotazioni tuple based}}, 2012.
      \item Co-Supervisor Francesco Cardi's bachelor thesis, 2012.
      \item Seminar ``From Engineer to Alchemist, There and Back Again: An Alchemist Tale'', 2012
      \item Seminar ``The simulation alchemy'', 2011
    \end{innerlist}
\item[] \textit{Contract Researcher} \hfill \textbf{June 2011 to December 2012}
    \begin{innerlist}
      \item Contract professor for the course ``\href{http://apice.unibo.it/xwiki/bin/view/Courses/SmaLm1112Lab}{Laboratory of Multi Agent Systems}'', 2011.
      \item Co-Supervisor in \href{http://apice.unibo.it/xwiki/bin/view/Theses/SapereComm}{Michele Morgagni's master thesis: \textit{Modulo di comunicazione in una infrastruttura per pervasive service ecosystems}}, 2011.
      \item Co-Supervisor in \href{http://www.alice.unibo.it/xwiki/bin/view/Theses/LSAspace}{Matteo Desanti's master thesis: \textit{Supporto a regole chimico-semantiche per la coordinazione di service pervasive ecosystems}}, 2011.
    \end{innerlist}
\halfblankline
\end{outerlist}

\href{https://www.bbs.unibo.eu/hp/}{\textbf{Bologna Business School}}, Bologna (BO), Italy
\begin{outerlist}
\item[] \textit{Professor} \hfill \textbf{October 2018 to December 2018}
    \begin{innerlist}
        \item \textit{Internet of Things -- Software production} --- advanced course on 
techniques for producing high quality software for the IoT. Focus on team coordination strategies 
and tools, build automation, testing, continuous integration, and continuous delivery
\end{innerlist}
\halfblankline
\end{outerlist}

\href{http://www.formart.it/}{\textbf{FORMart}}, Cesena (FC), Italy
\begin{outerlist}
\item[] \textit{Teacher} \hfill \textbf{January 2016 to March 2018}
    \begin{innerlist}
        \item \textit{``Programmazione e ICT problem solving'' --- course on algorithmic problem resolution and automation, with elements of programming in Python}
        \item \textit{``Sistemi informatici e loro gestione'' --- course on basics of operating systems, networking, and database management}
        \item \textit{``Elementi di Programmazione e Sviluppo di Applicazioni'' --- course on imperative and object oriented programming with C and Java}
    \end{innerlist}
\halfblankline
\end{outerlist}

\href{http://www.uiowa.edu/}{\textbf{University of Iowa}}, Iowa City, IA USA
\begin{outerlist}
 \item[] \textit{Visiting Researcher} \hfill \textbf{August 2014 to September 2014}
    \begin{innerlist}
      \item Seminar ``Programming Networks from the Aggregate Perspective''
    \end{innerlist}
\halfblankline
\end{outerlist}

\href{http://www.fit.edu/}{\textbf{Florida Institute of Technology}}, Melbourne, FL USA
\begin{outerlist}
 \item[] \textit{Visiting Researcher} \hfill \textbf{July 2009 to October 2009}
    \begin{innerlist}
      \item Seminar ``Self Organization in Coordination Systems using a Wordnet-based Ontology'', along with Sascia Virruso, under the supervision of Dr. Ronaldo Menezes
    \end{innerlist}
\halfblankline
\end{outerlist}

\section{Awards}
Best Paper Award, \href{https://saso2016.informatik.uni-augsburg.de/}{\textbf{SASO 2016}}, Augsburg, Germany

\section{International experience}
\href{http://www.uiowa.edu/}{\textbf{University of Iowa}}, Iowa City, IA USA
\begin{outerlist}
\item[] \textit{Visiting Researcher} \hfill \textbf{May 2016, to June 2016}
    \begin{innerlist}
      \item Advancements in the aggregate programming field.
      \item UIowa supervisor: Dr. Jacob Beal
      \item UniBo supervisor: Prof. Mirko Viroli
    \end{innerlist}
\halfblankline
\end{outerlist}

\href{http://www.uiowa.edu/}{\textbf{University of Iowa}}, Iowa City, IA USA
\begin{outerlist}
\item[] \textit{Visiting Researcher} \hfill \textbf{August 2014, to September 2014}
    \begin{innerlist}
      \item Research on aggregate programming and high order functions for field calculus. Refinement of Protelis.
      \item UIowa supervisor: Dr. Jacob Beal
      \item UniBo supervisor: Prof. Mirko Viroli
    \end{innerlist}
\halfblankline
\end{outerlist}

\href{http://www.bbn.com/}{\textbf{Raytheon BBN Technologies}}, Cambridge, MA USA
\begin{outerlist}
\item[] \textit{Visiting Researcher} \hfill \textbf{June 2014, to August 2014}
    \begin{innerlist}
      \item Research on aggregate programming and high order functions for field calculus. Realisation of Protelis.
      \item BBN supervisor: Dr. Jacob Beal
      \item UniBo supervisor: Prof. Mirko Viroli
    \end{innerlist}
\halfblankline
\end{outerlist}

\href{http://www.jku.at/}{\textbf{Johannes Kepler Universität}}, Linz, Austria
\begin{outerlist}
\item[] \textit{Visiting Researcher} \hfill \textbf{July 2013, to October 2013}
    \begin{innerlist}
      \item Research on crowd density estimation and prediction, crowd steering, crowd simulation, pervasive ecosystems.
      \item JKU supervisor: Univ.-Prof. Mag. Dr. Alois Ferscha
      \item UniBo supervisor: Prof. Mirko Viroli
    \end{innerlist}
\halfblankline
\end{outerlist}

\href{http://www.fit.edu/}{\textbf{Florida Institute of Technology}}, Melbourne, FL USA
\begin{outerlist}
\item[] \textit{Visiting Researcher} \hfill \textbf{July 2009 to October 2009}
    \begin{innerlist}
      \item Research on distributed systems, complex networks and self organisation
      \item FIT supervisor: Dr. Ronaldo Menezes
      \item UniBo supervisor: Prof. Andrea Omicini
    \end{innerlist}
\halfblankline
\end{outerlist}

\section{Professional Experience}
\href{http://www.twinlogix.com/en}{\textbf{twinlogix}}, Santarcangelo di Romagna (RN), Italy
\begin{outerlist}
\item[] \textit{Software development}%
        \hfill \textbf{March 2016 to January 2017}
\end{outerlist}
\halfblankline

\href{http://www.valpharma.com/}{\textbf{Valpharma International S.p.A.}}, Pennabilli (RN), Italy
\begin{outerlist}
\item[] \textit{Stage: raw material quality control}%
        \hfill \textbf{July 2004 to August 2004}
\end{outerlist}
\halfblankline

\section{Other activities}

Frequent contributor to \href{http://www.wikipedia.org/}{Wikipedia} and \href{http://www.openstreetmap.org/}{OpenStreetMap}.

\halfblankline{}

Creator and maintainer of the following Arch User Repository Packages, 2018--today
\begin{innerlist}
    \item \href{https://aur.archlinux.org/packages/opencorsairlink-git/}{\texttt{opencorsairlink-git}}.
    \item \href{https://aur.archlinux.org/packages/opencorsairlink-testing-git/}{\texttt{opencorsairlink-testing-git}}.
\end{innerlist}
\halfblankline

Contributor to \href{https://github.com/DanySK/Thread-Inheritable-Resource-Loader-for-Java}{Thread-inheritable resource loader for Java}, 2017--today
\begin{innerlist}
    \item A statically-usable resource and class loader that inherits the parent thread's class loader.
\end{innerlist}
\halfblankline

Contributor to \href{https://github.com/farkam135/GoIV}{GoIV}, 2017
\begin{innerlist}
    \item GoIV is an Android application devoted to rating the quality of Pokémon Go monsters relying solely on the on-screen information.
\end{innerlist}
\halfblankline

Contributor to \href{https://github.com/DanySK/gson-extras}{gson-extras}, 2017--today
\begin{innerlist}
    \item Extra libraries and component for Google Gson, extracted from the main repository and made publicly available on Maven Central.
\end{innerlist}
\halfblankline

Designer and developer of \href{https://github.com/DanySK/jirf}{JIRF}, 2017--today
\begin{innerlist}
    \item The Java Implicit Reflective Factory allows for building objects reflectively inside configured contexts, applying implicit type conversions chains when needed.
\end{innerlist}
\halfblankline

Designer and developer of \href{https://github.com/DanySK/urlclassloader-util}{URLClassloader Util}, 2016--today
\begin{innerlist}
    \item URLClassloader Util is a small library that provides functionality to manipulate the Java classpath at runtime.
\end{innerlist}
\halfblankline

Designer and developer of \href{https://github.com/DanySK/javadoc.io-linker}{Javadoc.io Linker}, 2016--today
\begin{innerlist}
    \item Javadoc.io linker is a Gradle plugin that configures any Javadoc build to link javadoc.io when referring to non-local classes.
\end{innerlist}
\halfblankline

Contributor to \href{https://github.com/Antergos/Cnchi}{Cnchi}, 2015
\begin{innerlist}
    \item Cnchi is a modern, flexible installer for Linux, developed by the Antergos Linux team.
\end{innerlist}
\halfblankline

Designer and leading developer of \href{http://protelis.org/}{Protelis}, 2014--today
\begin{innerlist}
    \item Protelis is a programming language aiming at making networked systems just as easy to build for complex and heterogeneous networks as for single machines and cloud systems. This accomplished by separating the different tasks and making some of the hard and subtle parts automatic and implicit.
\end{innerlist}
\halfblankline

Designer and leading developer of \href{http://alchemist.apice.unibo.it/}{Alchemist}, 2010--today
\begin{innerlist}
    \item Alchemist is an innovative simulator meant to join the expressiveness of the agent based modelling and the power and speed of the stochastic simulation algorithms used in chemistry. It is tailored to scenarios in which many nodes interact exchanging  informations. Its flexibility allows for a wide range of applications, spacing from the classical chemistry to the biology (e.g. complex morphogenesis processes) to pervasive computing.
\end{innerlist}
\halfblankline

Designer and developer of \href{https://github.com/DanySK/SmarTrRR}{SmarTrRR}, 2015--2017
\begin{innerlist}
    \item SmarTrRR is a transitive dependency range resolver plugin for Gradle. It replaces the default Protelis resolver, implementing a progressive range restriction, and a conflict resolution algorithm. Also, it allows the user to configure specific artifact substitutions.
\end{innerlist}
\halfblankline

Creator and maintainer of \href{https://bitbucket.org/danysk/nirvana-overlay/}{Nirvana overlay} for Gentoo Linux, 2014--2015
\begin{innerlist}
    \item Nirvana is an overlay for Gentoo Linux, namely a container of ebuild files, which are scripts describing how to install and maintain packages in a Gentoo Linux distribution. Nirvana contains those ebuild that work well, but are too hard to maintain to be pushed in Sunrise or Sabayon overlays. Moreover, this repository is used by me as a playground for creating new ebuilds. On July 2014 Nirvana got officially indexed by Layman, and as a consequence it is now available to all Gentoo users using such tool.
\end{innerlist}
\halfblankline

Creator and maintainer of {Nirvana Community Repository}, 2014--2015
\begin{innerlist}
    \item Nirvana Community Repository contains the same packages included in Nirvana overlay, distributed in a pre-compiled form compatible with Sabayon Linux Entropy package manager.
\end{innerlist}
\halfblankline

Designer and developer of \href{https://sourceforge.net/projects/mandelbrot/}{Angela the Mandelbrot Set Explorer}, 2009
\begin{innerlist}
  \item Angela is a Java parallel application that allows for visualizing portions of the Mandelbrot set.
\end{innerlist}
\halfblankline

Member of both the testing and development teams of \href{http://www.sabayon.org/}{Sabayon Linux}, 2008--2014
\begin{innerlist}
  \item Sabayon Linux is a Gentoo-based distribution which follows the works-out-of-the-box philosophy, aiming to give the user a wide number of applications that are ready for use and a self-configured operating system.
\end{innerlist}
\halfblankline

\href{http://www.astice.org/}{A.St.I.Ce. Executive Board Member}, January 2006 to November 2009
    \begin{innerlist}
      \item Founded ``I$^2$ --- Informa Ingegneri'', the technical journal of Seconda Facoltà di Ingegneria, containing articles about the research activity of the faculty.
      \item Founded ``Linux Libera Tutti'', a project meant to allow students access without any charge DVDs and CDs of various Linux distributions, with a special focus on Sabayon Linux.
    \end{innerlist}
\halfblankline

\section{Skills}

Computer Programming and software design:
\begin{innerlist}
    \item Java, Scala, Kotlin, C, Python, Prolog, Groovy, C$+$$+$, UNIX shell scripting, SQL, Xtend, and others.
    \item Language design with the Xtext framework
    \item Object Oriented design
    \item Distributed systems
    \item Concurrent programming
    \item Functional programming
    \item Mobile programming (Android)
\end{innerlist}
\halfblankline

Software engineering and productive teamwork:
\begin{innerlist}
    \item Distributed Version Control Systems (Mercurial, Git)
    \item Build systems (Gradle, Maven)
    \item Continuous Integration (Travis CI, drone.io)
    \item Automated software deployment
\end{innerlist}
\halfblankline

Hardware/software configuration:
\begin{innerlist}
    \item Windows installation and configuration
    \item Linux installation and configuration for personal computers, servers, and embedded systems with specific skills for Gentoo Linux, its derivatives and Arch Linux.
    \item Server and Desktop systems assembling
    \item Overclocking
\end{innerlist}
\halfblankline

Information/Internet Technology:
\begin{innerlist}
    \item Markup languages (XML, HTML, Markdown)
    \item Database manipulation with SQL
    \item Networking (UDP, TCP, ARP, DNS)
    \item Services (SQL, HTTP, application-specific daemon design)
    \item Content Managing (Joomla, Drupal)
    \item Static website generators (Jekyllrb)
\end{innerlist}
\halfblankline

Operating Systems:
\begin{innerlist}
    \item Linux, with specific skills for Gentoo, Sabayon and Arch
    \item other UNIX variants
    \item Microsoft Windows family
\end{innerlist}
\halfblankline

Productivity Applications:
\begin{innerlist}
    \item \LaTeX{}, \BibTeX{}
    \item Common productivity packages (for Windows and Linux platforms)
\end{innerlist}
\halfblankline

Multimedia (basic knowledge):
\begin{innerlist}
    \item Scalar image editing and analysis (Computer Vision skills, Photoshop, GIMP)
    \item Vectorial image editing (Dia, Inkscape)
    \item RAW image processing
    \item Non-linear video editing (Kdenlive, Openshot)
    \item 3D Design (Blender)
    \item 3D Programming (OpenGL)
\end{innerlist}


\section{Expertise}
Mathematics:
\begin{innerlist}
    \item Applied Mathematics, Real and Complex Analysis, Discrete Mathematics, Geometry.
\end{innerlist}
\halfblankline

Physics:
\begin{innerlist}
    \item Mechanics, Electromagnetism.
\end{innerlist}
\halfblankline

Control Theory and Engineering:
\begin{innerlist}
    \item Distributed and Self-adaptive Control, Dynamic Optimization, Bio-mimicry, Bio-inspiration.
\end{innerlist}
\halfblankline

Communications and Signal Processing:
\begin{innerlist}
    \item Probability, Random Variables, Stochastic Processes, Networks
\end{innerlist}
\halfblankline

Computer Science and Engineering:
\begin{innerlist}
    \item Model Checking, Software Verification, Component-Based Reusable Software, Object Oriented Programming, Logic Programming, Functional Programming, Concurrent Programming, Distributed Systems, Benchmarking, Model Driven Software Development.
\end{innerlist}
\halfblankline

Natural Sciences (Biology, Microbiology, Chemistry, Biochemistry):
\begin{innerlist}
    \item Molecular orbital theory, stoichiometry, organic chemistry, DNA transcription and replication processes, PCR, metabolic processes, virus classification, bacteria classification, human morphology, physiology, Earth sciences, astronomy.
\end{innerlist}

\section{References Available to Contact}
\href
{http://web.mit.edu/jakebeal/www/}
{\textbf{Dr. Jacob Beal}}
(e-mail:~\href{mailto:jakebeal@alum.mit.edu}{jakebeal@alum.mit.edu}; phone: +1 617 873 7676)
\begin{innerlist}
    \item Scientist,
        \href{http://www.bbn.com/}{Raytheon BBN Technologies}
    \item[$\diamond$] 10 Moulton Street, Cambridge, MA 02138, USA
    \item[$\star$] \emph{Dr. Beal was my local supervisor during my research period in Cambridge and Iowa City}
\end{innerlist}
\halfblankline

\href
{http://apice.unibo.it/xwiki/bin/view/AndreaOmicini/}
{\textbf{Prof. Andrea Omicini}}
(e-mail:~\href{mailto:andrea.omicini@unibo.it}{andrea.omicini@unibo.it}; phone: +39 0547 3 39220)
\begin{innerlist}
    \item Full Professor,
        \href{http://www.informatica.unibo.it/it}{Dipartimento di Informatica -- Scienza e Ingegneria}\\
        \href{http://www.unibo.it/}{Alma Mater Studiorum Università di Bologna}

    \item[$\diamond$] Via Venezia 52, 47521 Cesena (FC), Italy

    \item[$\star$] \emph{Prof. Omicini was my supervisor for Bachelor Thesis and my Italian supervisor during the research period in Florida Tech}
\end{innerlist}
\halfblankline

\href
{http://mirkoviroli.apice.unibo.it/}
{\textbf{Prof. Mirko Viroli}}
(e-mail:~\href{mailto:mirko.viroli@unibo.it}{mirko.viroli@unibo.it}; phone: +39 0547 3 39216)
\begin{innerlist}
    \item Associate Professor,
        \href{http://www.informatica.unibo.it/it}{Dipartimento di Informatica -- Scienza e Ingegneria}\\
        \href{http://www.unibo.it/}{Alma Mater Studiorum Università di Bologna}
    \item[$\diamond$] Via Venezia 52, 47521 Cesena (FC), Italy
    \item[$\star$] \emph{Dr. Viroli was my supervisor for Master Thesis and PhD}
\end{innerlist}
\halfblankline


\let\thefootnote\relax\footnotetext{\today}

\end{document}

%%%%%%%%%%%%%%%%%%%%%%%%%% End CV Document %%%%%%%%%%%%%%%%%%%%%%%%%%%%%

%----------------------------------------------------------------------%
% The following is copyright and licensing information for
% redistribution of this LaTeX source code; it also includes a liability
% statement. If this source code is not being redistributed to others,
% it may be omitted. It has no effect on the function of the above code.
%----------------------------------------------------------------------%
% Copyright (c) 2007, 2008, 2009, 2010, 2011 by Theodore P. Pavlic
%
% Unless otherwise expressly stated, this work is licensed under the
% Creative Commons Attribution-Noncommercial 3.0 United States License. To
% view a copy of this license, visit
% http://creativecommons.org/licenses/by-nc/3.0/us/ or send a letter to
% Creative Commons, 171 Second Street, Suite 300, San Francisco,
% California, 94105, USA.
%
% THE SOFTWARE IS PROVIDED "AS IS", WITHOUT WARRANTY OF ANY KIND, EXPRESS
% OR IMPLIED, INCLUDING BUT NOT LIMITED TO THE WARRANTIES OF
% MERCHANTABILITY, FITNESS FOR A PARTICULAR PURPOSE AND NONINFRINGEMENT.
% IN NO EVENT SHALL THE AUTHORS OR COPYRIGHT HOLDERS BE LIABLE FOR ANY
% CLAIM, DAMAGES OR OTHER LIABILITY, WHETHER IN AN ACTION OF CONTRACT,
% TORT OR OTHERWISE, ARISING FROM, OUT OF OR IN CONNECTION WITH THE
% SOFTWARE OR THE USE OR OTHER DEALINGS IN THE SOFTWARE.
%----------------------------------------------------------------------%
