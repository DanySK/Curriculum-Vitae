%%%%%%%%%%%%%%%%%%%%%%%%%%%%%%%%%%%%%%%%%%%%%%%%%%%%%%%%%%%%%%%%%%%%%%%%
%%%%%%%%%%%%%%%%%%%%%% Simple LaTeX CV Template %%%%%%%%%%%%%%%%%%%%%%%%
%%%%%%%%%%%%%%%%%%%%%%%%%%%%%%%%%%%%%%%%%%%%%%%%%%%%%%%%%%%%%%%%%%%%%%%%

%%%%%%%%%%%%%%%%%%%%%%%%%%%%%%%%%%%%%%%%%%%%%%%%%%%%%%%%%%%%%%%%%%%%%%%%
%% NOTE: If you find that it says                                     %%
%%                                                                    %%
%%                           1 of ??                                  %%
%%                                                                    %%
%% at the bottom of your first page, this means that the AUX file     %%
%% was not available when you ran LaTeX on this source. Simply RERUN  %%
%% LaTeX to get the ``??'' replaced with the number of the last page  %%
%% of the document. The AUX file will be generated on the first run   %%
%% of LaTeX and used on the second run to fill in all of the          %%
%% references.                                                        %%
%%%%%%%%%%%%%%%%%%%%%%%%%%%%%%%%%%%%%%%%%%%%%%%%%%%%%%%%%%%%%%%%%%%%%%%%

%%%%%%%%%%%%%%%%%%%%%%%%%%%% Document Setup %%%%%%%%%%%%%%%%%%%%%%%%%%%%

% Don't like 10pt? Try 11pt or 12pt
\documentclass[10pt]{article}
\immediate\write18{./setup_ruby.sh}
\immediate\write18{./scholar_scraper.rb}
\newcommand{\sapere}{\texttt{SAPERE}}
\newcommand{\doititle}[2]{\href{http://dx.doi.org/#1}{#2}}

% This is a helpful package that puts math inside length specifications
\usepackage{calc}
\usepackage[utf8]{inputenc} % Consente l'uso caratteri accentati italiani
\usepackage{xcolor}
\usepackage{tabularx}
\usepackage[hyphens]{url}
\usepackage[
backend=biber,
style=ieee,
sorting=ynt,
url=false,
defernumbers=true,
maxbibnames=99,
bibencoding=utf8
]{biblatex}
\usepackage[hidelinks]{hyperref}
\usepackage{underscore}
\usepackage[normalem]{ulem}
% \usepackage{lmodern}
\usepackage[T1]{fontenc}
\usepackage{textcomp}
\hypersetup{
    colorlinks,
    breaklinks,
    linkcolor=darkblue,
    urlcolor=darkblue,
    anchorcolor=darkblue,
    citecolor=darkblue
}
\addbibresource{bibliography.bib}

% Layout: Puts the section titles on left side of page
\reversemarginpar

%
%         PAPER SIZE, PAGE NUMBER, AND DOCUMENT LAYOUT NOTES:
%
% The next \usepackage line changes the layout for CV style section
% headings as marginal notes. It also sets up the paper size as either
% letter or A4. By default, letter was used. If A4 paper is desired,
% comment out the letterpaper lines and uncomment the a4paper lines.
%
% As you can see, the margin widths and section title widths can be
% easily adjusted.
%
% ALSO: Notice that the includefoot option can be commented OUT in order
% to put the PAGE NUMBER *IN* the bottom margin. This will make the
% effective text area larger.
%
% IF YOU WISH TO REMOVE THE ``of LASTPAGE'' next to each page number,
% see the note about the +LP and -LP lines below. Comment out the +LP
% and uncomment the -LP.
%
% IF YOU WISH TO REMOVE PAGE NUMBERS, be sure that the includefoot line
% is uncommented and ALSO uncomment the \pagestyle{empty} a few lines
% below.
%

\usepackage[paper=a4paper,
           %includefoot, % Uncomment to put page number above margin
           marginparwidth=30.5mm,    % Length of section titles
           marginparsep=1.5mm,       % Space between titles and text
           margin=25mm,              % 25mm margins
           includemp]{geometry}

%% More layout: Get rid of indenting throughout entire document
\setlength{\parindent}{0in}

%% This gives us fun enumeration environments. compactitem will be nice.
\usepackage{paralist}

%% Reference the last page in the page number
%
% NOTE: comment the +LP line and uncomment the -LP line to have page
%       numbers without the ``of ##'' last page reference)
%
% NOTE: uncomment the \pagestyle{empty} line to get rid of all page
%       numbers (make sure includefoot is commented out above)
%
\usepackage{fancyhdr,lastpage}
\pagestyle{fancy}
%\pagestyle{empty}      % Uncomment this to get rid of page numbers
\fancyhf{}\renewcommand{\headrulewidth}{0pt}
\fancyfootoffset{\marginparsep+\marginparwidth}
\newlength{\footpageshift}
\setlength{\footpageshift}
          {0.5\textwidth+0.5\marginparsep+0.5\marginparwidth-2in}
\lfoot{\hspace{\footpageshift}%
       \parbox{4in}{\, \hfill %
                    \arabic{page} of \protect\pageref*{LastPage} % +LP
%                    \arabic{page}                               % -LP
                    \hfill \,}}

% Finally, give us PDF bookmarks
\usepackage{color}
\definecolor{darkblue}{rgb}{0.0,0.0,0.3}


%%%%%%%%%%%%%%%%%%%%%%%% End Document Setup %%%%%%%%%%%%%%%%%%%%%%%%%%%%


%%%%%%%%%%%%%%%%%%%%%%%%%%% Helper Commands %%%%%%%%%%%%%%%%%%%%%%%%%%%%

% The title (name) with a horizontal rule under it
% (optional argument typesets an object right-justified across from name
%  as well)
%
% Usage: \makeheading{name}
%        OR
%        \makeheading[right_object]{name}
%
% Place at top of document. It should be the first thing.
% If ``right_object'' is provided in the square-braced optional
% argument, it will be right justified on the same line as ``name'' at
% the top of the CV. For example:
%
%       \makeheading[\emph{Curriculum vitae}]{Your Name}
%
% will put an emphasized ``Curriculum vitae'' at the top of the document
% as a title. Likewise, a picture could be included:
%
%   \makeheading[\includegraphics[height=1.5in]{my_picutre}]{Your Name}
%
% the picture will be flush right across from the name.
\newcommand{\makeheading}[2][]%
        {\hspace*{-\marginparsep minus \marginparwidth}%
         \begin{minipage}[t]{\textwidth+\marginparwidth+\marginparsep}%
             {\large \bfseries #2 \hfill #1}\\[-0.15\baselineskip]%
                 \rule{\columnwidth}{1pt}%
         \end{minipage}}

% The section headings
%
% Usage: \section{section name}
%
% Follow this section IMMEDIATELY with the first line of the section
% text. Do not put whitespace in between. That is, do this:
%
%       \section{My Information}
%       Here is my information.
%
% and NOT this:
%
%       \section{My Information}
%
%       Here is my information.
%
% Otherwise the top of the section header will not line up with the top
% of the section. Of course, using a single comment character (%) on
% empty lines allows for the function of the first example with the
% readability of the second example.
\renewcommand{\section}[2]%
        {\pagebreak[3]\vspace{1.3\baselineskip}%
         \phantomsection\addcontentsline{toc}{section}{#1}%
         \hspace{0in}%
         \marginpar{
         \raggedright \scshape #1}#2}

% An itemize-style list with lots of space between items
\newenvironment{outerlist}[1][\enskip\textbullet]%
        {\begin{itemize}[#1]}{\end{itemize}%
         \vspace{-.6\baselineskip}}

% An environment IDENTICAL to outerlist that has better pre-list spacing
% when used as the first thing in a \section
\newenvironment{lonelist}[1][\enskip\textbullet]%
        {\vspace{-\baselineskip}\begin{list}{#1}{%
        \setlength{\partopsep}{0pt}%
        \setlength{\topsep}{0pt}}}
        {\end{list}\vspace{-.6\baselineskip}}

% An itemize-style list with little space between items
\newenvironment{innerlist}[1][\enskip\textbullet]%
        {\begin{compactitem}[#1]}{\end{compactitem}}

% An environment IDENTICAL to innerlist that has better pre-list spacing
% when used as the first thing in a \section
\newenvironment{loneinnerlist}[1][\enskip\textbullet]%
        {\vspace{-\baselineskip}\begin{compactitem}[#1]}
        {\end{compactitem}\vspace{-.6\baselineskip}}

% To add some paragraph space between lines.
% This also tells LaTeX to preferably break a page on one of these gaps
% if there is a needed pagebreak nearby.
\newcommand{\blankline}{\quad\pagebreak[3]}
\newcommand{\halfblankline}{\quad\vspace{-0.5\baselineskip}\pagebreak[3]}

% Uses hyperref to link DOI
\newcommand\doilink[1]{\href{http://dx.doi.org/#1}{#1}}
\newcommand\doi[1]{doi:\doilink{#1}}

% For \url{SOME_URL}, links SOME_URL to the url SOME_URL
\providecommand*\url[1]{\href{#1}{#1}}
% Same as above, but pretty-prints SOME_URL in teletype fixed-width font
\renewcommand*\url[1]{\href{#1}{\texttt{#1}}}

% For \email{ADDRESS}, links ADDRESS to the url mailto:ADDRESS
\providecommand*\email[1]{\href{mailto:#1}{#1}}
% Same as above, but pretty-prints ADDRESS in teletype fixed-width font
%\renewcommand*\email[1]{\href{mailto:#1}{\texttt{#1}}}

%\providecommand\BibTeX{{\rm B\kern-.05em{\sc i\kern-.025em b}\kern-.08em
%    T\kern-.1667em\lower.7ex\hbox{E}\kern-.125emX}}
%\providecommand\BibTeX{{\rm B\kern-.05em{\sc i\kern-.025em b}\kern-.08em
%    \TeX}}
\providecommand\BibTeX{{B\kern-.05em{\sc i\kern-.025em b}\kern-.08em
    \TeX}}
\providecommand\Matlab{\textsc{Matlab}}

%%%%%%%%%%%%%%%%%%%%%%%% End Helper Commands %%%%%%%%%%%%%%%%%%%%%%%%%%%

%%%%%%%%%%%%%%%%%%%%%%%%% Begin CV Document %%%%%%%%%%%%%%%%%%%%%%%%%%%%

\newcommand{\macrosection}[1]{

\vspace{20pt}
\hrulefill
\begin{center}
\textbf{#1}
\end{center}
\vspace{-7pt}
\hrulefill
}

\begin{document}
\makeheading{Danilo Pianini}

\section{Contacts}
%
% NOTE: Mind where the & separators and \\ breaks are in the following
%       table.
%
% ALSO: \rcollength is the width of the right column of the table
%       (adjust it to your liking; default is 1.85in).
%
\newlength{\rcollength}\setlength{\rcollength}{2.5in}%
%
\begin{tabular}[t]{@{}p{\textwidth-\rcollength}p{\rcollength}}
%\href{http://www.cse.osu.edu/}%
%     {Department of Computer Science and Engineering} & \\
%\href{http://www.osu.edu/}{The Ohio State University}

Via dell'Università, 50          & \textit{Landline:} +39 0547 33 88 20\\
47522 Cesena (FC)          & \textit{E-mail:} \email{danilo.pianini@unibo.it}\\
Italy                      & \textit{WWW:} \href{http://www.danilopianini.org/}{www.danilopianini.org}\\
\end{tabular}

\section{Citizenship}
Italy

\section{Current placement}
\textbf{University of Bologna, Department of Computer Science and Engineering} \hfill
\begin{outerlist}
    \item Senior assistant professor (fixed-term)\\\emph{Ricercatore a tempo determinato tipo b) (senior)}
\end{outerlist}

\section{Research Themes}
%
My current research is focused on the engineering aspects of pervasive computing, with the goal of providing a robust, easy, and coherent chain of tools and procedures that can lead to robust, adaptive, self-healing, and possibly evolving software ecosystems.
%
I enjoy exploring and developing novel programming paradigms and languages, which change the way we think about and model application domains.

\newpage

\macrosection{EDUCATION AND CARREER}

\section{Education}
%
\href{http://www.deis.unibo.it/DEIS/default.htm}{\textbf{Dipartimento di Ingegneria Elettronica, Informatica e delle Telecomunicazioni, Università di Bologna}}, Bologna (BO), Italy
\begin{outerlist}
\item[] Ph.D. in
        \href{http://www.cse.unibo.it/en/phd-program}
             {Electronics, Computer Science and Telecommunications Engineering},
             April 2015
        \begin{innerlist}
        \item Thesis Title: \href{http://amsdottorato.unibo.it/7000/}{\emph{Engineering Complex Computational Ecosystems}}
%         \item Thesis Proposal: \emph{Cooperative Task Processing}
%         \item Candidacy: \emph{Research
%             Problems in Distributed Control for Energy Systems}
        \item Supervisor:
              \href{http://mirkoviroli.apice.unibo.it/}{Prof. Mirko Viroli}
        \item Tutor:
              \href{http://lia.deis.unibo.it/Staff/AntonioNatali/}{Prof. Antonio Natali}
        \item External reviewer:
              \href{http://cui.unige.ch/~dimarzo/}{Prof. Giovanna di Marzo Serugendo}
        \item External reviewer:
              \href{http://www.simondobson.org/}{Prof. Simon Dobson}
        \item Area of Study: Pervasive computing
        \end{innerlist}

\halfblankline{}
\end{outerlist}
\href{http://www.ing2.unibo.it/Ingegneria+Cesena/default.htm}{\textbf{Seconda Facoltà di Ingegneria, Università di Bologna}}, Cesena (FC), Italy
\begin{outerlist}
\item[] M.S.,
        \href{http://www.ing2.unibo.it/Ingegneria+Cesena/default.htm}
             {Computer Engineering}, March 2011
        \begin{innerlist}
        \item \emph{110L/110 - Magna cum Laude}
        \item Thesis Topic: \emph{A Framework for the Simulation of Pervasive Services Ecosystems}
        \item Supervisor:
              \href{http://mirkoviroli.apice.unibo.it/}
                   {Prof. Mirko Viroli}
        \item Area of Study: Computational Models
        \end{innerlist}

\item[] B.S.,
        \href{http://www.ing2.unibo.it/Ingegneria+Cesena/default.htm}
             {Computer Engineering}, October 2008
        \begin{innerlist}
        \item Thesis Topic: \emph{From Swarm Intelligence to Self-Organising Coordination: a Pervasive Scenarios Application}
        \item Supervisor:
              \href{http://apice.unibo.it/xwiki/bin/view/AndreaOmicini/}
                   {Prof. Andrea Omicini}
        \item Area of Study: Distributed Systems
        \end{innerlist}

\halfblankline
\end{outerlist}
\href{http://www.iiseinaudi.it/}{\textbf{ITCG L. Einaudi}}, Novafeltria (RN), Italy
\begin{outerlist}
 \item [ ] Scientific high school, focus on biology, July 2005
        \begin{innerlist}
	  \item 100/100
        \end{innerlist}
\end{outerlist}


\section{Certifications}
\href{http://www.miur.gov.it/abilitazione-scientifica-nazionale}{Abilitazione Scientifica Nazionale al ruolo di professore di I fascia}\\
Code 09/H1 --- Information processing systems \\
Italian Ministry of Education, Universities and Research\\
\href{https://asn23.cineca.it/pubblico/miur/esito-abilitato/09%252FH1/1/2}{Starting 2024-11-19, ending 2035-11-19}

\halfblankline

\href{http://www.miur.gov.it/abilitazione-scientifica-nazionale}{Abilitazione Scientifica Nazionale al ruolo di professore di II fascia}\\
Code 01/B1 --- Informatics \\
Italian Ministry of Education, Universities and Research\\
\href{https://asn18.cineca.it/pubblico/miur/esito-abilitato/01%252FB1/2/5}{Starting 2020-11-23, ending 2029-11-23}

\halfblankline

\href{http://www.miur.gov.it/abilitazione-scientifica-nazionale}{Abilitazione Scientifica Nazionale al ruolo di professore di II fascia}\\
Code 09/H1 --- Information processing systems \\
Italian Ministry of Education, Universities and Research\\
\href{https://asn16.cineca.it/pubblico/miur/esito-abilitato/09%252FH1/2/5}{Starting 2018-07-26, ending 2027-07-26}


\section{Research contracts}
% ! TeX root = curriculum.tex
\textbf{Senior assistant professor @ University of Bologna} \hfill \textbf{2022-10-10 --- 2025-10-09}\\
Italian \emph{Ricercatore a tempo determinato tipo b) (senior)}, three years.
Tenure track.

\halfblankline{}

\textbf{Junior assistant professor @ University of Bologna} \hfill \textbf{2019-12-19 --- 2022-10-09}\\
Italian \emph{Ricercatore a tempo determinato tipo a) (junior)}, three years.
Fixed term research and teaching position,
within the scope of the Fluidware PRIN project.

\halfblankline{}

\textbf{Research contract @ University of Bologna} \hfill \textbf{2019-01-01 --- 2019-12-18}\\
Italian \emph{Assegno di ricerca}, 12 months.
The research contract I was working on has been extended for a further year.

\halfblankline{}

\textbf{Research contract @ University of Bologna} \hfill \textbf{2018-01-01 --- 2018-12-31}\\
Italian \emph{Assegno di ricerca}, 12 months.
The research contract I was working on has been extended for a further year.

\halfblankline{}

\textbf{Research contract @ University of Bologna} \hfill \textbf{2017-01-01 --- 2017-12-31}\\
Italian \emph{Assegno di ricerca}, 12 months.
Research contract: ``Engineering collective adaptive systems through aggregate computing''.

\halfblankline{}

\textbf{Research contract @ University of Bologna} \hfill \textbf{2016-01-01 --- 2016-12-31}\\
Italian \emph{Assegno di ricerca}, 12 months.
Research contract: ``Platforms and tools for aggregate programming''.

\halfblankline{}

\textbf{Research contract @ University of Bologna} \hfill \textbf{2015-01-01 --- 2015-12-31}\\
Italian \emph{Assegno di ricerca}, 12 months.
Research contract: ``Prototyping a tool-chain for engineering self-organising systems by aggregate programming''.

\halfblankline{}

\textbf{Ph.D. student @ University of Bologna} \hfill \textbf{2012-01-01 --- 2015-04-15}\\
Ph.D. course covered by the victory of a grant from the Italian Ministry of Research.
I further work with the SAPERE consortium at the beginning, then I began exploring novel research lines, contributing to the birth of aggregate programming.

\halfblankline{}

\textbf{Research contract @ University of Bologna} \hfill \textbf{2011-05-01 --- 2011-12-12}\\
Italian \emph{Assegno di ricerca}, 12 months.
I worked within the SAPERE Project, developing the toolchain.
The contract was interrupted before its natural conclusion as incompatible with the Ph.D. grant.


\newpage

\macrosection{SCIENTIFIC PUBLICATIONS}

\section{Bibliometrics}
\input{scholar}

\vspace{0.1in}
\section{Journal publications sorted by time}
\renewcommand{\section}[2]{}
\nocite{*}
\vspace{-0.29in}
\DeclareFieldFormat{labelnumberwidth}{#1}
\printbibliography[type=article,omitnumbers=true]{}

\renewcommand{\section}[2]%
        {\pagebreak[3]\vspace{1.3\baselineskip}%
         \phantomsection\addcontentsline{toc}{section}{#1}%
         \hspace{0in}%
         \marginpar{
         \raggedright \scshape #1}#2}
\section{Contributions in conference proceedings sorted by time}
\renewcommand{\section}[2]{}
\vspace{-0.29in}
\defbibfilter{conferences}{
  type=inproceedings or
  type=incollection
}
\printbibliography[filter=conferences,omitnumbers=true]{}

\renewcommand{\section}[2]%
        {\pagebreak[3]\vspace{1.3\baselineskip}%
         \phantomsection\addcontentsline{toc}{section}{#1}%
         \hspace{0in}%
         \marginpar{
         \raggedright \scshape #1}#2}
\section{Other publications (books, proceedings, chapters, etc) sorted by time}
\renewcommand{\section}[2]{}
\vspace{-0.29in}
\defbibfilter{other}{
  not ( type=inproceedings or type=incollection or type=article )
}
\printbibliography[filter=other,omitnumbers=true]{}

\newpage

\renewcommand{\section}[2]%
        {\pagebreak[3]\vspace{1.3\baselineskip}%
         \phantomsection\addcontentsline{toc}{section}{#1}%
         \hspace{0in}%
         \marginpar{
         \raggedright \scshape #1}#2}

\macrosection{SCIENTIFIC ACTIVITIES}

\section{Participation in research groups}
% ! TeX root = curriculum.tex
\textbf{APICe Research group} \hfill \textbf{March 2011 --- today}\\
The primary research group I worked with since the conclusion of my two years master. The group includes professors Andrea Omicini, Mirko Viroli, Antonio Natali, Alessandro Ricci, and operates at the University of Bologna.

\halfblankline{}

\textbf{DISI/Raytheon BBN Technologies} \hfill \textbf{November 2013 --- today}\\
Collaboration between my research unit and the research team in Raytheon BBN Technologies led by dr. Jacob Beal. The collaboration fostered the birth of aggregate computing. A decisive step forward was taken in 2014, when I visited for three months, designing and developing the Protelis programming language. I visited again dr. Jacob Beal at the University of Iowa in 2016. The collaboration recently led to a research contract of \$40.000 between BBN and UniBo.

\halfblankline{}

\textbf{CommonWears consortium} \hfill \textbf{September 2022 --- today}\\
Consortium of five italian universities, PRIN 2022 Italian project.
\begin{innerlist}
    \item Università della Calabria
    \item Alma Mater Studiorum---Università di Bologna
    \item Free University of Bozen-Bolzano
    \item Università Campus Bio-Medico di Roma
    \item Università di Torino
\end{innerlist}
Among others, the consortium includes professors Giancarlo Fortino, Ferruccio Damiani, and Antonio Liotta.

\halfblankline{}

\textbf{Aggregate computing research group} \hfill \textbf{2015 --- today}\\
International collaboration between:
\begin{innerlist}
    \item Alma Mater Studiorum---Università di Bologna;
    \item Raytheon BBN Technologies; and
    \item Università di Torino.
\end{innerlist}
The research group is envisioning, developing, and testing methods and technologies for conceiving systems with a focus on their global behavior. The UniBo team is composed by a subset of the APICe Research group, while the two other teams are led dr. Jacob Beal and prof. Ferruccio Damiani.

\halfblankline{}

\textbf{FluidWare consortium} \hfill \textbf{September 2019 --- 2023}\\
Consortium of four italian universities, PRIN 2017 Italian project.
\begin{innerlist}
    \item Università degli studi di Modena e Reggio Emilia
    \item Alma Mater Studiorum---Università di Bologna
    \item Università di Camerino
    \item Università della Calabria
\end{innerlist}
Among others, the consortium includes professors Franco Zambonelli, Giancarlo Fortino and Barbara Re.

\halfblankline{}

\textbf{SAPERE Consortium} \hfill \textbf{March 2011 --- September 2013}\\
International consortium of five universities, within which I worked during the first half of my PhD.
\begin{innerlist}
    \item Università degli studi di Modena e Reggio Emilia
    \item Alma Mater Studiorum---Università di Bologna
    \item Johannes Kepler Universität Linz
    \item University of St Andrews
    \item Université de Genève
\end{innerlist}
The participation allowed me to work with professors Franco Zambonelli, Marco Mamei, Alois Ferscha, Simon Dobson, and Giovanna Di Marzo Serugendo, and their research groups.

\halfblankline{}

\textbf{Institut für Pervasive Computing} \hfill \textbf{July 2013 --- October 2013}\\
Research group coordinated by prof. Alois Ferscha. I had been working with the team for three months, focussing on analysis of pedestrian data to prevent crowd disasters.

\halfblankline{}

\textbf{Distributed systems @ Florida Tech} \hfill \textbf{July 2009 --- October 2009}\\
My first research activity has been carried on with prof. Ronaldo Menezes at the Florida Institute of Technologies while I was still a master student. I joined his research team for three months, studying the coordination of distributed systems and complex networks.

\halfblankline{}


\section{Research projects}
\textit{WOOD4.0 - Woodworking Machines for Industry 4.0}\\
\begin{innerlist}
    \item \textbf{Type:} Emilia--Romagna regional project
    \item \textbf{Consortium:} SCM Group S.p.A. (leader), Alma Mater Studiorum---Università di Bologna
    \item \textbf{Role:} Unit leader, scientific principal investigator
    \item \textbf{Duration:} January 2023--December 2025 (expected)
    \item \textbf{Budget (UniBo unit):} €300K UniBo unit
    \item \textbf{Reference:} art. 6. LR 14/2014 Emilia-Romagna – DGR 1098/2022 tipologia B (PG/2022/1032336), CUP: E69J22007520009
\end{innerlist}

\section{Scientific responsibility of research contracts}
\textit{
    Studio, realizzazione, e validazione di un prototipo di Piattaforma
    Digitale interconnessa per l’aggiornamento del software da remoto per
    macchine industriali
}
\begin{innerlist}
    \item Angelo Filaseta \hfill{} July 2023---Ongoing
\end{innerlist}

\halfblankline

\textit{
    Sistemi, strumenti, e metodologie per il deployment e l’aggiornamento di software industriale nel cloud-edge continuum
}
\begin{innerlist}
    \item Martina Baiardi \hfill{} August 2023---October 2023
\end{innerlist}

\section{Supervision of PhD students}
\vspace{-1em}
\begin{innerlist}
\item Angela Cortecchia \hfill{} November 2024---Ongoing
\item Davide Domini \hfill{} November 2023---Ongoing
\item Nicolas Farabegoli \hfill{} November 2023---Ongoing
\item Martina Baiardi \hfill{} November 2023---Ongoing
\item Ruslan Shaiakhmetov \hfill{} November 2022---Ongoing
\end{innerlist}

\section{Editorial activity}
% ! TeX root = curriculum.tex
\href{https://www.mdpi.com/journal/ijgi/}{\textbf{International Journal of Geo-Information}}: journal topic board member --- 2019 -- today
\\ \halfblankline \\
\href{http://blog.ieeesoftware.org/}{\textbf{IEEE Software Blog}}: associate blog editor --- 2019 -- today
\\ \halfblankline \\
\href{https://www.hindawi.com/journals/sp/}{\textbf{Scientific programming}}: academic editor --- 2017 -- today
\\ \halfblankline \\
\href{https://www.springer.com/gp/book/9783030053321}{\textbf{The Future of Digital Democracy --- An Interdisciplinary Approach}} (ISBN 978-3-030-05333-8): editor -- 2019


% !TEX root = curriculum.tex
\section{{\color{black}Chairing in international conferences}}
\halfblankline \\
\href{https://conf.researchr.org/home/acsos-2022/}{3rd International Conference on Autonomic Computing and Self-Organizing Systems
(ACSOS 2022)}
\\ Special Event Chair \\
\halfblankline \\
\href{https://conf.researchr.org/home/acsos-2021}{2nd IEEE International Conference on Autonomic Computing and Self-Organizing Systems 
(ACSOS 2021)}
\\ Program Committee chair \\
\halfblankline \\
\href{https://ngps2019.github.io/}{Next Generation Programming Languages and Systems (NGPS 2019) --- 
Track of the 34th ACM Symposium on Applied Computing (SAC 2019)}
\\ Track chair \\
\halfblankline \\
\href{https://saso2018.fbk.eu/}{12th IEEE International Conference on Self-Adaptive and Self-Organizing (SASO 2018)}
\\ Workshops and tutorials chair \\
\halfblankline \\
\href{http://icac2018.informatik.uni-wuerzburg.de/committees/organization-committee/}{15th IEEE International Conference on Autonomic Computing (ICAC 2018)}
\\ Workshops and tutorials chair \\
\halfblankline \\
\href{http://sac-cas2018.apice.unibo.it/referees.html}{Collective and Cooperative Systems --- Special Track of the 33rd ACM Symposium on Applied Computing (SAC 2018)}
\\ Session chair \\
\halfblankline \\
\href{http://apice.unibo.it/xwiki/bin/view/ALP4IoT2016/WebHome}{1st workshop on Architectures, Languages and Paradigms for IoT (ALP4IoT 2017)}
\\ Program Committee chair \\

\section{{\color{black}Participation in panels}}
\halfblankline \\
\href{https://2022.acsos.org/details/acsos-2022-papers/36/Hot-Topics-and-Current-Trends-in-ACSOS-Research}{3rd International Conference on Autonomic Computing and Self-Organizing Systems
    (ACSOS 2022)}
\\ \href{https://danysk.github.io/Slides-2022-ACSOS-Panel/#/}{Hot Topics and Current Trends in ACSOS Research} \\

\section{{\color{black}Participation in program committees of international conferences}}
\halfblankline \\
\href{https://icaart.scitevents.org/}{15th International Conference on Agents and Artificial Intelligence
(ICAART 2023)}
\\ Program Committee member \\
\halfblankline \\
\href{https://sissy.telecom-paristech.fr/}{Workshop on Self-Improving Systems Integration (SISSY 2022)}
\\ Program Committee member \\
\halfblankline \\
\href{https://conf.researchr.org/home/acsos-2022/}{3rd International Conference on Autonomic Computing and Self-Organizing Systems
(ACSOS 2022)}
\\ Program Committee member \\
\halfblankline \\
\href{https://ifm22.si.usi.ch/}{17th International Conference on integrated Formal Methods
(iFM 2022)}
\\ Artifact Evaluation Committee member \\
\halfblankline \\
\href{https://discoli-workshop.github.io/2022/}{1st Workshop on DIStributed COLlective Intelligence
(DISCOLI 2022)}
\\ Program Committee member \\
\halfblankline \\
\href{https://www.discotec.org/2022/coordination}{1st Workshop on Adaptive, Learning PervAsive Computing Applications
(ALPACA 2022)}
\\ Program Committee member \\
\halfblankline \\
\href{https://www.discotec.org/2022/coordination}{24th International Conference on Coordination Models and Languages 
(COORDINATION 2022)}
\\ Artefact Evaluation Committee member \\
\halfblankline \\
\href{http://www.icaart.org/?y=2022}{14th International Conference on Agents and Artificial Intelligence 
(ICAART 2022)}
\\ Program Committee member \\
\halfblankline \\
\href{http://www.icaart.org/?y=2021}{13th International Conference on Agents and Artificial Intelligence 
(ICAART 2021)}
\\ Program Committee member \\
\halfblankline \\
\href{http://archive.vn/wip/38Ah6}{5th Workshop on Engineering Collective Adaptive Systems (eCAS 2020)}
\\ Program Committee member \\
\halfblankline \\
\href{https://conf.researchr.org/home/acsos-2020}{1st IEEE International Conference on Autonomic Computing and Self-Organizing Systems (ACSOS 2020)}
\\ Program Committee member \\
\halfblankline \\
\href{https://https://ijcai20.org/}{29th International Joint Conference on Artificial Intelligence (IJCAI 2020)}
\\ Program Committee member \\
\halfblankline \\
\href{https://aamas2020.conference.auckland.ac.nz/program-committee-members/}{International Conference on Autonomous Agents and Multi-Agent Systems 2020 (AAMAS 2020)}
\\ Program Committee member \\
\halfblankline \\
\href{http://www.discotec.org/2020/coordination}{COORDINATION 2020 - 22nd International Conference on Coordination Models and Languages}
\\ Program Committee member \\
\halfblankline \\
\href{http://www.icaart.org/?y=2020}{12th International Conference on Agents and Artificial Intelligence 
(ICAART 2020)}
\\ Program Committee member \\
\halfblankline \\
\href{http://apice.unibo.it/xwiki/bin/view/ECAS2019/Committees}{4th Workshop on Engineering Collective Adaptive Systems, (eCAS 2019)}
\\ Program Committee member \\
\halfblankline \\
\href{http://aamas2019.encs.concordia.ca/}{International Conference on Autonomous Agents and 
Multiagent Systems (AAMAS 2019)}
\\ Program Committee member \\
\halfblankline \\
\href{http://www.icaart.org/?y=2019}{11th International Conference on Agents and Artificial Intelligence 
(ICAART 2019)}
\\ Program Committee member \\
\halfblankline \\
\href{http://www.discotec.org/2019/coordination}{COORDINATION 2019 - 21st International Conference on Coordination Models and Languages}
\\ Program Committee member \\
\halfblankline \\
\href{http://diid.unipa.it/roboticslab/woa2018/}{XIX Workshop "From Objects to Agents" (WOA 2018)}
\\ Program Committee member \\
\halfblankline \\
\href{http://sac-cas2018.apice.unibo.it/referees.html}{Collective and Cooperative Systems --- Special Track of the 33rd ACM Symposium on Applied Computing (SAC 2018)}
\\ Program Committee member \\
\halfblankline \\
\href{http://apice.unibo.it/xwiki/bin/view/ECAS2017/WebHome}{2nd eCAS Workshop on Engineering Collective Adaptive Systems (eCAS 2017)}
\\ Program Committee member \\
\halfblankline \\
\href{http://woa2017.unirc.it/}{XVIII WORKSHOP "From Objects to Agents" (WOA 2017)}
\\ Program Committee member \\


\section{Reviewing for international journals}
% ! TeX root = curriculum.tex
\href{https://www.sciencedirect.com/journal/science-of-computer-programming}{\textbf{Science of Computer Programming} (ISSN: 01676423)}, 2022
\\ \halfblankline \\
\href{https://www.computer.org/csdl/journal/sc}{\textbf{IEEE Transactions on Services Computing} (ISSN: 19391374)}, 2022
\\ \halfblankline \\
\href{https://www.iospress.com/catalog/journals/intelligenza-artificiale}{\textbf{IOS Press Intelligenza Artificiale} (ISSN: 17248035, 22110097)}, 2022
\\ \halfblankline \\
\href{https://www.journals.elsevier.com/expert-systems-with-applications}{\textbf{Elsevier Expert Systems With Applications} (ISSN: 09574174)}, 2021
\\ \halfblankline \\
\href{https://www.mdpi.com/journal/network}{\textbf{MDPI Network} (ISSN: 2673-8732)}, 2021
\\ \halfblankline \\
\href{https://www.journals.elsevier.com/journal-of-information-security-and-applications}{\textbf{Elsevier Journal of Information Security and Applications} (ISSN: 22142126, 22142134)}, 2021
\\ \halfblankline \\
\href{https://www.mdpi.com/journal/electronics}{\textbf{MDPI Electronics} (ISSN: 20799292)}, 2020-2021
\\ \halfblankline \\
\href{https://www.sciencedirect.com/journal/future-generation-computer-systems}{\textbf{Elsevier Future Generation Computer Systems} (ISSN: 0167739X)}, 2020--2023
\\ \halfblankline \\
\href{https://www.journals.elsevier.com/pervasive-and-mobile-computing}{\textbf{Elsevier Pervasive and Mobile Computing} (ISSN: 15741192)}, 2020
\\ \halfblankline \\
\href{http://www.mdpi.com/journal/sensors}{\textbf{MDPI Sensors} (ISSN: 14243210, 14248220)}, 2016-2020
\\ \halfblankline \\
\href{https://link.springer.com/journal/10462}{\textbf{Springer Artificial Intelligence Review (AIRE)} (ISSN: 02692821, 15737462)}, 2019
\\ \halfblankline \\
\href{https://www.hindawi.com/journals/mpe/}{\textbf{Hindawi Mathematical Problems in Engineering} (ISSN: 1024123X, 15635147)}, 2017--2019
\\ \halfblankline \\
\href{http://www.mdpi.com/journal/applsci}{\textbf{MDPI Applied sciences} (ISSN: 20763417)}, 2018
\\ \halfblankline \\
\href{https://academic.oup.com/comjnl}{\textbf{Oxford University Press The Computer Journal} (ISSN: 00104620, 14602067)}, 2018
\\ \halfblankline \\
\href{https://www.journals.elsevier.com/artificial-intelligence-in-medicine/}{\textbf{Elsevier Artificial Intelligence in Medicine} (ISSN: 09333657, 18732860)}, 2018
\\ \halfblankline \\
\href{https://www.journals.elsevier.com/computational-and-structural-biotechnology-journal/}{\textbf{Elsevier Computational and Structural Biotechnology Journal} (ISSN: 20010370)}, 2016
\\ \halfblankline \\
\href{http://cacm.acm.org/}{\textbf{Communications of the ACM} (ISSN: 00010782, 15577317)}, 2016


\section{Invited talks in international conferences}
% ! TeX root = curriculum.tex
\href{https://danysk.github.io/slides-2024-acsos-reproducibility/}{\sout{Picture} Reproducible experiments, or didn’t happen} \\
\href{https://2024.acsos.org/}{\textit{Invited Doctoral Symposium Talk, 5th IEEE International Conference on Autonomic Computing and Self-Organizing Systems (ACSOS 2024)}}
\\ \halfblankline \\


\section{Talks in international conferences}
% ! TeX root = curriculum.tex
\href{https://danysk.github.io/slides-2023-asmecc/}{Infrastructures for the Edge-Cloud Continuum on a Small Scale: a Practical Case Study} \\
\href{https://asmecc-workshop.github.io/2023/}{\textit{1st ASMECC Workshop on Autonomic and Self-* Management for the Edge-Cloud Continuum (ASMECC 2023)}}
\\ \halfblankline \\
\href{https://danysk.github.io/slides-2023-dais-loadshift/}{Runtime Load-Shifting of Distributed Controllers Across Networked Devices} \\
\href{https://www.discotec.org/2023/dais.html}{\textit{23rd International Conference on Distributed Applications and Interoperable Systems (DAIS 2023)}}
\\ \halfblankline \\
\href{https://danysk.github.io/Slides-2022-ACSOS-BoundedElection/}{Self-stabilising Priority-Based Multi-Leader Election and Network Partitioning} \\
\href{https://2022.acsos.org/}{\textit{3rd IEEE International Conference on Autonomic Computing and Self-Organizing Systems - ACSOS 2022}}
\\ \halfblankline \\
\href{https://danysk.github.io/Slides-2022-Coordination-SpaceFluid/}{Space-Fluid Adaptive Sampling: a Field-Based, Self-Organising Approach} \\
\href{https://danysk.github.io/Slides-2022-Coordination-SpaceFluid/}{\textit{24st International Conference on Coordination Models and Languages (COORDINATION 2022)}}
\\ \halfblankline \\
\href{https://alchemistsimulator.github.io/tutorials/basics/index.html}{Simulation of Large Scale Computational Ecosystems with Alchemist: A Tutorial} \\
\href{https://www.discotec.org/2023/dais.html}{\textit{21st International Conference on Distributed Applications and Interoperable Systems (DAIS 2021)}}
\\ \halfblankline \\
\href{https://danysk.github.io/Slides-2020-Coordination-TimeFluid/}{Time-Fluid Field-Based Coordination} \\
\href{http://www.discotec.org/2020/coordination.html}{\textit{22nd International Conference on Coordination Models and Languages (COORDINATION 2020)}}
\\ \halfblankline \\
\href{https://danysk.github.io/Slides-2019-Coordination-SCR/}{Self-organising Coordination Regions: a pattern for edge computing} \\
\href{http://www.discotec.org/2019/coordination.html}{\textit{21st International Conference on Coordination Models and Languages (COORDINATION 2019)}}
\\ \halfblankline \\
\href{https://danysk.github.io/Slides-2019-eCAS-security/}{Security in Collective Adaptive Systems: a Roadmap} \\
\href{https://apice.unibo.it/xwiki/bin/view/ECAS2019/}{\textit{4th eCAS Workshop on Engineering Collective Adaptive Systems (eCAS 2019)}}
\\ \halfblankline \\
\href{https://danysk.github.io/Slides-2018-BISS/}{Computing at the Aggregate Level} \\
\href{https://www.biss-institute.com/wp-content/uploads/2018/07/For-more-information-download-the-brochure.pdf}{\textit{Workshop ``Making the smart city safe for citizens:
The case of smart energy and mobility''}}
\\ \halfblankline \\
\href{https://www.slideshare.net/DanySK/engineering-the-aggregate-talk-at-software-engineering-for-intelligent-and-autonomous-systems-sefias-dagstuhl-2018}{Engineering the Aggregate} \\
\href{https://www.hpi.uni-potsdam.de/giese/public/selfadapt/dagstuhl-seminars/sefias/}{\textit{GI Dagstuhl Seminar ``Software Engineering for Intelligent and Autonomous Systems'' (SEfIAS 2018)}}
\\ \halfblankline \\
Themes and Challenges in Engineering CAS \\
Panelist at the \href{http://apice.unibo.it/xwiki/bin/view/ECAS2017/WebHome}{\textit{2nd eCAS Workshop on Engineering Collective Adaptive Systems (eCAS 2017)}}
\\ \halfblankline \\
\href{https://www.slideshare.net/DanySK/practical-aggregate-programming-with-protelis-saso2017}{Practical Aggregate Programming with Protelis} \\
Tutorial at the \href{http://apice.unibo.it/xwiki/bin/view/ECAS2017/WebHome}{\textit{11th IEEE International Conference on Self-Adaptive and Self-Organizing Systems (SASO 2017)}}
\\ \halfblankline \\
\href{https://www.slideshare.net/DanySK/2nd-ecas-workshop-on-engineering-collective-adaptive-systems}{Towards a Foundational API for Resilient Distributed Systems Design} \\
\href{http://apice.unibo.it/xwiki/bin/view/ECAS2017/WebHome}{\textit{2nd eCAS Workshop on Engineering Collective Adaptive Systems (eCAS 2017)}}
\\ \halfblankline \\
\href{https://www.slideshare.net/DanySK/simulating-largescale-aggregate-mass-with-alchemist-and-scala}{Simulating Large-scale Aggregate MASs with Alchemist and Scala} \\
\href{https://fedcsis.org/2016/mass}{\textit{10th International Workshop on Multi-Agent Systems and Simulation (MAS\&S 2016)}}
\\ \halfblankline \\
\href{https://www.slideshare.net/DanySK/computational-fields-meet-augmented-reality-perspectives-and-challenges}{Computational Fields meet Augmented Reality: Perspectives and Challenges} \\
\href{http://www.spatial-computing.org/scopes}{\textit{1st Workshop on Spatial and COllective PErvasive Computing Systems (SCOPES 2015)}}
\\ \halfblankline \\
\href{http://apice.unibo.it/xwiki/bin/download/PRIMA2015/Demos/Pianini.pdf}{Engineering multi-agent systems with aggregate computing} \\
Demo at the \href{http://apice.unibo.it/xwiki/bin/view/PRIMA2015/Demos}{\textit{18th Conference on Principles and Practice of Multi-Agent Systems (PRIMA 2015)}}
\\ \halfblankline \\
\href{https://www.slideshare.net/DanySK/extending-the-gillespies-stochastic-simulation-algorithm-for-integrating-discreteevent-and-multiagent-based-simulation}{Extending the Gillespie's Stochastic Simulation Algorithm for Integrating Discrete-Event and Multi-Agent Based Simulation} \\
\href{http://www.springer.com/gp/book/9783319314464}{\textit{XVI International Workshop on Multi-Agent Based Simulation (MABS 2015)}}
\\ \halfblankline \\
\href{https://www.slideshare.net/DanySK/sac-47194849}{Protelis: Practical Aggregate Programming} \\
\href{https://www.sigapp.org/sac/sac2015/}{\textit{The 30th ACM/SIGAPP Symposium On Applied Computing (SAC 2015)}}
\\ \halfblankline \\
\href{https://www.slideshare.net/DanySK/gradientbased-selforganisation-patterns-of-anticipative-adaptation}{Gradient-based Self-organisation Patterns of Anticipative Adaptation} \\
\href{http://saso2012.univ-lyon1.fr/}{\textit{6th IEEE International Conference on Self-Adaptive and Self-Organizing Systems (SASO 2012)}}
\\ \halfblankline \\
\href{http://apice.unibo.it/xwiki/bin/view/Talks/PianiniMass2011}{A Chemical Inspired Simulation Framework for Pervasive Services Ecosystems} \\
\href{https://fedcsis.org/2011/}{\textit{5th International Workshop on Multi-Agent Systems and Simulation (MAS\&S 2011)}}
\\ \halfblankline \\
\href{http://apice.unibo.it/xwiki/bin/view/Talks/PianiniVirrusoSASO10}{Self Organization in Coordination Systems using a WordNet-based Ontology} \\
\href{http://www.inf.u-szeged.hu/projectdirs/saso10/}{\textit{Fourth IEEE International Conference on Self-Adaptive and Self-Organizing Systems (SASO 2010)}}


\section{Other talks}
\href{https://danysk.github.io/Slides-2019-OYM/}{From Nature Inspiration to Aggregate Computing} \\
Seminar for ``Open Your Mind'', 2019
\\ \halfblankline \\
\href{https://www.slideshare.net/DanySK/continuous-integration-and-delivery-77394349}{Continuous integration and delivery} \\
Seminar for the ``\href{http://apice.unibo.it/xwiki/bin/view/Courses/PPS1617}{Programming and development paradigms}'' course, 2016
\\ \halfblankline \\
\href{https://www.slideshare.net/DanySK/democratic-process-and-electronic-platforms-concerns-of-an-engineer}{Democratic process and electronic platforms: concerns of an engineer} \\
Workshop ``\href{https://sites.google.com/site/futureofdemocracy2016/home/workshop}{The Future of Democracy}'', 2016
\\ \halfblankline \\
\href{https://www.slideshare.net/DanySK/software-development-made-serious}{Software development made serious} \\
Seminar for the ``\href{http://apice.unibo.it/xwiki/bin/view/Courses/ISAC1516}{Adaptive complex software systems engineering}'' course, 2016
\\ \halfblankline \\
\href{https://www.slideshare.net/DanySK/engineering-complex-computational-ecosystems-phd-defense}{Engineering Complex Computational Ecosystems} \\
PhD defense, 2015
\\ \halfblankline \\
\href{}{Programming Networks from the Aggregate Perspective} \\
Seminar, University of Iowa, Iowa City (IA), USA, 2014
\\ \halfblankline \\
\href{https://www.slideshare.net/DanySK/engineering-computational-ecosystems-2nd-year-ph-d-seminar}{Engineering computational ecosystems} \\
2nd year PhD seminar, 2013
\\ \halfblankline \\
\href{http://campus.unibo.it/115195/}{From Engineer to Alchemist, There and Back Again: An Alchemist Tale} \\
Seminar for the ``\href{http://apice.unibo.it/xwiki/bin/view/Courses/LsaLm1213}{Laboratory of systems and applications LM}'' course, 2012
\\ \halfblankline \\
\href{https://www.slideshare.net/DanySK/engineering-computational-ecosystems}{Engineering computational ecosystems} \\
Vieni via con noi, 2012, Cesena
\\ \halfblankline \\
\href{https://www.slideshare.net/DanySK/recipes-for-sabayon-cook-your-own-linux-distro-within-two-hours}{Recipes for Sabayon: cook your own Linux distro within two hours} \\
Linux Day 2012, Cesena
\\ \halfblankline \\
\href{http://campus.unibo.it/83921/}{The simulation alchemy} \\
Seminar for ``\href{http://apice.unibo.it/xwiki/bin/view/Courses/LsaLm1112}{Laboratory of systems and applications LM}'', 2011
\\ \halfblankline \\
\href{http://apice.unibo.it/xwiki/bin/view/Talks/PianiniWoa2011}{A Simulation Framework for Pervasive Service Ecosystems} \\
\href{http://www.inf.u-szeged.hu/projectdirs/saso10/}{\textit{XII Workshop ``Dagli Oggetti agli Agenti''} (WOA 2010)}
\\ \halfblankline \\
\href{}{Self Organization in Coordination Systems using a Wordnet-based Ontology} \\
Seminar, Florida Institute of Technology, Melbourne (FL), USA, 2009

\section{Awards}
\textbf{Best Poster Award}\\
\href{https://2024.acsos.org/}{
  \textbf{5th IEEE International Conference on Autonomic Computing and Self-Organizing Systems (ACSOS 2024)},
    Aarhus, Denmark.
}\\
``\href{https://github.com/DanySK/poster-2024-acsos-imageonomics-drones/blob/f426784e6d6daa98f2288d298d059617231b6104/poster-printed.pdf}{Decentralized Multi-Drone Coordination for Wildlife Video Acquisition}''

\halfblankline\\
\textbf{Best Paper Award}\\
\href{https://emas.in.tu-clausthal.de/2024/}{
  \textbf{12th International Workshop on Engineering Multi-Agent Systems (EMAS 2024)},
  Auckland, New Zealand.\
}\\
``\href{https://doi.org/10.48550/arXiv.2404.10397}{On the external concurrency of current BDI frameworks for MAS}''

\halfblankline\\
\textbf{Best Paper Award}\\
\href{https://saso2016.informatik.uni-augsburg.de/}{\textbf{2016 IEEE 10th International Conference on Self-Adaptive and Self-Organizing Systems (SASO 2016)}}, Augsburg, Germany.\\
``\href{https://ieeexplore.ieee.org/document/7774387}{Self-Adaptation to Device distribution Changes}''
has been selected as the best contribution among the
\href{https://ieeexplore.ieee.org/xpl/conhome/7774239/proceeding?rowsPerPage=50&pageNumber=1}{20 papers that got past the peer review}.
The award included an invitation to submit an extended version to the
\href{https://dl.acm.org/journal/taas}{ACM Transactions on Autonomous and Adaptive Systems},
a reference journals for academics and practitioners in self-organization.

\section{International experience}
\href{http://www.uiowa.edu/}{\textbf{University of Iowa}}, Iowa City, IA USA
\begin{outerlist}
\item[] \textit{Visiting Researcher} \hfill \textbf{May 2016, to June 2016}
    \begin{innerlist}
      \item Advancements in the aggregate programming field.
      \item UIowa supervisor: Dr. Jacob Beal
      \item UniBo supervisor: Prof. Mirko Viroli
    \end{innerlist}
\halfblankline
\end{outerlist}

\href{http://www.uiowa.edu/}{\textbf{University of Iowa}}, Iowa City, IA USA
\begin{outerlist}
\item[] \textit{Visiting Researcher} \hfill \textbf{August 2014, to September 2014}
    \begin{innerlist}
      \item Research on aggregate programming and high order functions for field calculus. Refinement of Protelis.
      \item UIowa supervisor: Dr. Jacob Beal
      \item UniBo supervisor: Prof. Mirko Viroli
    \end{innerlist}
\halfblankline
\end{outerlist}

\href{http://www.bbn.com/}{\textbf{Raytheon BBN Technologies}}, Cambridge, MA USA
\begin{outerlist}
\item[] \textit{Visiting Researcher} \hfill \textbf{June 2014, to August 2014}
    \begin{innerlist}
      \item Research on aggregate programming and high order functions for field calculus. Realisation of Protelis.
      \item BBN supervisor: Dr. Jacob Beal
      \item UniBo supervisor: Prof. Mirko Viroli
    \end{innerlist}
\halfblankline
\end{outerlist}

\href{http://www.jku.at/}{\textbf{Johannes Kepler Universität}}, Linz, Austria
\begin{outerlist}
\item[] \textit{Visiting Researcher} \hfill \textbf{July 2013, to October 2013}
    \begin{innerlist}
      \item Research on crowd density estimation and prediction, crowd steering, crowd simulation, pervasive ecosystems.
      \item JKU supervisor: Univ.-Prof. Mag. Dr. Alois Ferscha
      \item UniBo supervisor: Prof. Mirko Viroli
    \end{innerlist}
\halfblankline
\end{outerlist}

\href{http://www.fit.edu/}{\textbf{Florida Institute of Technology}}, Melbourne, FL USA
\begin{outerlist}
\item[] \textit{Visiting Researcher} \hfill \textbf{July 2009 to October 2009}
    \begin{innerlist}
      \item Research on distributed systems, complex networks and self organisation
      \item FIT supervisor: Dr. Ronaldo Menezes
      \item UniBo supervisor: Prof. Andrea Omicini
    \end{innerlist}
\halfblankline
\end{outerlist}

\newpage
\macrosection{TEACHING}

% ! TeX root = curriculum.tex

{
\scriptsize{
\textbf{NOTE:} In most courses in Italy students are mandatorily submitted an anonymous form where they can express their opinion about the course in a four-valued scale (very negative, negative, positive, very positive). Such evaluations are reported here when available, in the form $(A, B, C, D, E)$, where:
\\$A$: Overall course satisfaction;
\\$B$: Availability of the teacher;
\\$C$: Clarity of exposition;
\\$D$: The teacher stimulates learning interest;
\\$E$: Number of respondents.
\\With the exception of $E$, the fraction of students evaluating positively or very positively is reported.
}
}

\newcommand{\shortcourse}[6]{
    \href{#1}{\textit{\textbf{#2}}},
    \href{#3}{#4},
    \href{#5}{#6}}

\newcommand{\course}[9]{
    \shortcourse{#1}{#2}{#3}{#4}{#5}{#6},
    #7,
    #8.
    \textit{#9}
}

\newcommand{\shortunibocourse}[4]{
    \shortcourse{#1}{#2}
    {http://www.unibo.it}{Alma Mater Studiorum---Università di Bologna}
    {#3}{#4}}

\newcommand{\unibocourse}[7]{
    \course{#1}{#2}
    {http://www.unibo.it}{Alma Mater Studiorum---Università di Bologna}
    {#3}{#4}
    {#5}
    {#6}{#7}
}

\section{PhD courses}
\vspace{-2em}
\begin{outerlist}
    \item[2021/22]
        \unibocourse
        {https://danysk.github.io/phd-course-devops-science/}
        {DevOps meets scientific research}
        {https://archive.ph/pH7Hk}
        {PhD program in Computer Science and Engineering }
        {June 2023}
        {20 hours}
        {Application of DevOps methodologies to scientific artifacts, reproducibility, sharing, licensing, versioning.}
    \item[2019/20]
        \unibocourse
        {https://bitbucket.org/danysk/courses-2018-developing-maintaining-and-sharing-software-tools/downloads/}
        {Developing, Maintaining, and Sharing Software Tools for Research}
        {http://archive.fo/Fmxxm}
        {PhD program in Data Science and Computation}
        {October 2019}
        {8 hours}
        {Creation of shareable and reproducible research software through DevOps methodologies; licensing.}
    \item[2019/20]
        \unibocourse
        {http://archive.fo/HKAC6/}
        {DevOps for scientific research}
        {http://archive.fo/ppTiB}
        {Phd Program in Computer Science and Engineering}
        {September 2019}
        {20 hours}
        {Creation of shareable and reproducible research software through DevOps tools and methodologies; versioning; licensing.}
    \item[2018/19]
        \unibocourse
        {https://bitbucket.org/danysk/courses-2018-developing-maintaining-and-sharing-software-tools/downloads/}
        {Developing, Maintaining, and Sharing Software Tools for Research}
        {http://archive.fo/Fmxxm}
        {PhD program in Data Science and Computation}
        {2018}
        {20 hours}
        {Creation of shareable and reproducible research software through DevOps tools and methodologies; versioning; licensing.}
\end{outerlist}

\section{Course Responsibility}
\vspace{-2em}
\begin{outerlist}
    \item[2021/22]
        \unibocourse
        {https://archive.ph/75yeO}{Laboratory of Software Systems (Laboratorio di Sistemi Software---Titolare del corso)}
        {https://archive.ph/ljEQn}{Two year master in Computer Science and Engineering (Laurea magistrale in Ingegneria e Scienze informatiche)}
        {3 CFU/ECTS, 30 hours, \textasciitilde{}30 students}
        {(76.5\%, 88.2\%, 76.5\%, 88.2\%, 17)}
        {Internal DSLs in Kotlin and their use in build systems, DevOps practices, build automation (Gradle as paradigmatic automator), continuous integration, advanced version control and team organization, licensing, open source sharing.}
    \item[2020/21]
        \unibocourse
        {https://archive.is/CyIjt}{Laboratory of Software Systems (Laboratorio di Sistemi Software---Titolare del corso)}
        {https://archive.is/Svmye}{Two year master in Computer Science and Engineering (Laurea magistrale in Ingegneria e Scienze informatiche)}
        {3 CFU/ECTS, 30 hours, \textasciitilde{}30 students}
        {(96\%, 96\%, 95.8\%, 100\%, 25)}
        {Internal DSLs in Kotlin and their use in build systems, DevOps practices, build automation (Gradle as paradigmatic automator), continuous integration, advanced version control and team organization, licensing, open source sharing.}
\end{outerlist}

\section{Teaching of learning modules}
\vspace{-1.9em}
\newcommand{\oopjava}{Practical OOP programming in Java from simple examples to advanced mechanisms and design patterns, including parallel and functional-like programming; basics of quality assurance, profiling, and build automation; distributed version control.}
\newcommand{\sedtmeit}{DevOps: Principles, practices, and tools; agile development and SCRUM; testing; continuous integration and delivery.}
\begin{outerlist}
    \item[2021/22]
        \unibocourse %DTM
        {https://archive.ph/OxWCT}{Software Engineering}
        {https://archive.ph/fs8a6}{Two year master in Digital Transformation Management}
        {2 CFU/ECTS, 18 hours}
        {(90.9\%, 100\%, 100\%, 100\%, 22)}
        {
            \sedtmeit{}
            Note: this course module is shared with ``Computer Science and Engineering''.
        }
    \item[2021/22]
        \unibocourse %EIT
        {https://archive.ph/1mBPz}{Software Engineering for Intelligent Distributed Systems}
        {https://archive.ph/ljEQn}{Two year master in Computer Science and Engineering}
        {2 CFU/ECTS, 18 hours}
        {(90.9\%, 100\%, 100\%, 100\%, 22)}
        {
            \sedtmeit{}
            Note: this course module is shared with ``Digital Transformation Management''.
        }
    \item[2021/22]
        \unibocourse
        {https://archive.ph/oUCdx}{Object-Oriented Programming (Programmazione ad Oggetti)}
        {https://archive.ph/vKgXj}{Bachelor in Computer Science and Engineering (Laurea in Ingegneria e scienze informatiche)}
        {3 CFU/ECTS, 30 hours, \textasciitilde{}150 students}
        {(91.1\%, 99.1\%, 93.6\%, 92.8\%, 112)}
        {\oopjava}
    \item[2020/21]
        \unibocourse
        {https://archive.is/WxwNn}{Object-Oriented Programming (Programmazione ad Oggetti)}
        {https://archive.is/WeNI1}{Bachelor in Computer Science and Engineering (Laurea in Ingegneria e scienze informatiche)}
        {3 CFU/ECTS, 30 hours, \textasciitilde{}150 students}
        {(77.7\%, 96.4\%, 79.5\%, 79.5\%, 112)}
        {\oopjava}
    \item[2019/20]
        \unibocourse
        {http://archive.fo/JtEDW}{Object-Oriented Programming (Programmazione ad Oggetti)}
        {http://archive.fo/UM5wl}{Bachelor in Computer Science and Engineering (Laurea in Ingegneria e scienze informatiche)}
        {3 CFU/ECTS, 30 hours, \textasciitilde{}150 students}
        {(83.6\%, 99.2\%, 86.0\%, 90.8\%, 130)}
        {\oopjava}
    \item[2018/19]
        \unibocourse
        {http://archive.fo/srdtN}{Object-Oriented Programming (Programmazione ad Oggetti)}
        {http://archive.fo/UM5wl}{Bachelor in Computer Science and Engineering (Laurea in Ingegneria e scienze informatiche)}
        {3 CFU/ECTS, 30 hours, \textasciitilde{}150 students}
        {(91.3\%, 99.0\%, 91.3\%, 91.3\%, 104)}
        {\oopjava}
    \item[2017/18]
        \unibocourse
        {http://archive.fo/54lT9}{Object-Oriented Programming (Programmazione ad Oggetti)}
        {http://archive.fo/UM5wl}{Bachelor in Computer Science and Engineering (Laurea in Ingegneria e scienze informatiche)}
        {3 CFU/ECTS, 30 hours, \textasciitilde{}150 students}
        {(87.9\%, 99.1\%, 83.5\%, 94.0\%, 116)}
        {\oopjava}
    \item[2016/17]
        \unibocourse
        {http://archive.fo/0XLbd}{Object-Oriented Programming (Programmazione ad Oggetti)}
        {http://archive.fo/UM5wl}{Bachelor in Computer Science and Engineering (Laurea in Ingegneria e scienze informatiche)}
        {3 CFU/ECTS, 30 hours, \textasciitilde{}150 students}
        {(82.1\%, 97.5\%, 95.1\%, 90.2\%, 123)}
        {\oopjava}
    \item[2015/16]
        \unibocourse
        {http://archive.fo/1eIMt}{
            Engineering Complex Adaptive Software Systems
            (Ingegneria dei Sistemi Software Adattativi Complessi)
        }
        {http://archive.fo/toz5c}{Two-year master in Computer Science and Engineering (Laurea magistrale in Ingegneria e Scienze informatiche)}
        {1 CFU/ECTS, 9 hours, \textasciitilde{}10 students}
        {(100\%, 100\%, 100\%, 100\%, 6)}
        {Distributed self-organization using programmable tuple spaces; simulation and reproducibility.}
    \item[2014/15]
        \unibocourse
        {http://archive.fo/JrWEu/}{Foundations of Informatics A (Fondamenti di Informatica A)}
        {http://archive.fo/30rN0}{Bachelor in Electronics Engineering for Energy and Information (Laurea in Ingegneria elettronica)}
        {3 CFU/ECTS, 30 hours, \textasciitilde{}75 students}
        {(88.7\%, 95.1\%, 98.4\%, 98.4\%, 62)}
        {
            Practical imperative programming in C, from foundations to graphical interfaces.
            Note: this course module is shared with ``Bachelor in Biomedical Engineering''.
        }
    \item[2014/15]
        \unibocourse
        {http://archive.fo/JrWEu/}{Foundations of Informatics A (Fondamenti di Informatica A)}
        {http://archive.fo/jW52L}{Bachelor in Biomedical Engineering (Laurea in Ingegneria biomedica)}
        {3 CFU/ECTS, 30 hours, \textasciitilde{}75 students}
        {(88.7\%, 95.1\%, 98.4\%, 98.4\%, 62)}
        {
            Practical imperative programming in C, from foundations to graphical interfaces.
            Note: this course module is shared with ``Bachelor in Electronics Engineering for Energy and Information''.
        }
    \item[2011/12]
        \unibocourse
        {http://archive.fo/nnsBl/}
        {Multi-Agent Systems (Sistemi multi-agente LM)}
        {http://archive.fo/qDVq3}{Two-year master in Computer Engineering (Laurea magistrale in Ingegneria Informatica)}
        {3 CFU/ECTS, 30 hours, \textasciitilde{}15 students}
        {(90.9\%, 100\%, 100\%, 100\%, 11)}
        {Practical multi-agent design and programming; distributed multi-agent platforms (Jade), innovative paradigms and programming languages (Jason).}
\end{outerlist}

\section{Tutoring}
\vspace{-1.9em}
\begin{outerlist}
    \item[2019/20]
        \shortunibocourse
        {http://archive.fo/JtEDW}{Object-Oriented Programming}
        {http://archive.fo/UM5wl}{Bachelor in Computer Science and Engineering}, \textasciitilde{}150 students.
    \item[2018/19]
        \shortunibocourse
        {http://archive.fo/srdtN}{Object-Oriented Programming}
        {http://archive.fo/UM5wl}{Bachelor in Computer Science and Engineering}, \textasciitilde{}150 students.
    \item[2015/16]
        \shortunibocourse
        {http://archive.fo/puTDG}{Object-Oriented Programming}
        {http://archive.fo/UM5wl}{Bachelor in Computer Science and Engineering}, \textasciitilde{}150 students.
    \item[2014/15]
        \shortunibocourse
        {http://archive.fo/5LhhW}{Engineering Complex Adaptive Software Systems}
        {http://archive.fo/toz5c}{Two-year master in Computer Science and Engineering}.
    \item[2014/15]
        \shortunibocourse
        {http://archive.fo/8jzEp}{Object-Oriented Programming}
        {http://archive.fo/UM5wl}{Bachelor in Computer Science and Engineering}, \textasciitilde{}150 students.
    \item[2013/14]
        \shortunibocourse
        {http://archive.fo/h8JCD}{Engineering Complex Adaptive Software Systems}
        {http://archive.fo/toz5c}{Two-year master in Computer Science and Engineering}.
    \item[2013/14]
        \shortunibocourse
        {http://archive.fo/0Gr16}{Object-Oriented Programming}
        {http://archive.fo/UM5wl}{Bachelor in Computer Science and Engineering}, \textasciitilde{}150 students.
    \item[2013/14]
        \shortunibocourse
        {http://archive.fo/XZFR0}{Foundations of Informatics A}
        {http://archive.fo/UM5wl}{Bachelor in Electronics Engineering for Energy and Information, Bachelor in Biomedical Engineering}, \textasciitilde{}75 students.
\end{outerlist}

\section{Supervision of graduate students}
\vspace{-1em}
\begin{innerlist}
    \item Kyrillos Ntronov: \href{https://amslaurea.unibo.it/29647/}{\textit{Comparative Benchmarking of Multithreading Solutions for JVM Languages: the case of the Alchemist Simulator}}, 2023.
    \item Mirko Felice: \href{https://amslaurea.unibo.it/29136/}{\textit{Progettazione e sviluppo di un'API dichiarativa per il testing di plugin Gradle}}, 2023.
    \item Gianmarco Magnani: \href{https://amslaurea.unibo.it/28184/}{\textit{Integrazione del simulatore Alchemist con il tool statistico MultiVeStA}}, 2023.
    \item Davide Schiaroli: \href{https://amslaurea.unibo.it/28132/}{\textit{Controllo di accesso a risorse di cluster Kubernetes tramite Active Directory}}, 2023.
    \item Angelo Filaseta: \href{https://amslaurea.unibo.it/28111/}{\textit{Distributed monitoring and control with dynamic offloading: the case of the Alchemist Simulator}}, 2023.
    \item Nicolas Farabegoli: \href{https://amslaurea.unibo.it/28035/}{\textit{Design and Implementation of a Portable Framework for Application Decomposition and Deployment in Edge-Cloud Systems}}, 2023.
    \item Elisa Tronetti: \href{https://amslaurea.unibo.it/28077/}{\textit{Towards Aggregate Programming in pure Kotlin through compiler-level metaprogramming}}, 2023.
    \item Alessandro Neri: \href{http://amslaurea.unibo.it/23043/}{\textit{Da software monolitico a DevOps e Microservizi: un caso di studio industriale}}, 2021.
    \item Loris Cangini: \href{http://amslaurea.unibo.it/20410/}{\textit{Una piattaforma client-server universale per Aggregate Computing}}, 2020.
    \item Niccolò Maltoni: \href{http://amslaurea.unibo.it/20478/}{\textit{Progettazione di una piattaforma web per la simulazione di programmi aggregati}}, 2020.
    \item Andrea Placuzzi: \href{http://amslaurea.unibo.it/20484/}{\textit{A platform for aggregate computing over LoRaWAN network}}, 2020.
    \item Giacomo Scaparrotti: \href{http://amslaurea.unibo.it/20440/}{\textit{Cross-simulator integration: ns3 as a network simulation back-end for Alchemist}}, 2020.
    \item Filippo Nicolini: \href{http://amslaurea.unibo.it/19521/}{\textit{Simulazione di Agenti BDI basati su Prolog in Alchemist}}, 2019.
    \item Matteo Francia: \href{http://amslaurea.unibo.it/13090/}{\textit{A Foundational Library for Aggregate Programming}}, 2017.
    \item Simone Costanzi: \href{http://amslaurea.unibo.it/10519/}{\textit{Integrazione di piattaforme d'esecuzione e simulazione in una toolchain Scala per aggregate programming}}, 2016.
    \item Davide Ensini: \href{http://amslaurea.unibo.it/7990/}{\textit{Spatial computing per smart devices}}, 2014.
    \item Luca Nenni: \href{http://amslaurea.unibo.it/6927/}{\textit{Simulazioni realistiche di algoritmi di Crowd Steering}}, 2014.
    \item Enrico Polverelli: \href{http://amslaurea.unibo.it/5293/}{\textit{Simulazione di algoritmi di auto-organizzazione basati su gradiente computazionale in Alchemist}}, 2012.
    \item Andrea Dallatana: \href{http://amslaurea.unibo.it/4217/}{\textit{BDI agents for Real Time Strategy games}}, 2012.
    \item Francesca Cioffi: \href{http://amslaurea.unibo.it/4088/}{\textit{Algoritmi gradient-based per la modellazione e simulazione di sistemi auto-organizzanti}}, 2012.
    \item Paolo Contessi: \href{http://amslaurea.unibo.it/4074/}{\textit{Supporting semantic web technologies in the pervasive service ecosystems middleware}}, 2012.
    \item Giacomo Pronti: \href{http://archive.fo/nBeOg}{\textit{Simulazione di ecosistemi di servizi pervasivi con supporto ad annotazioni tuple based}}, 2012.
    \item Michele Morgagni: \href{http://archive.fo/6mnSN}{\textit{Modulo di comunicazione in una infrastruttura per pervasive service ecosystems}}, 2011.
    \item Matteo Desanti: \href{http://archive.fo/rwla1}{\textit{Supporto a regole chimico-semantiche per la coordinazione di service pervasive ecosystems}}, 2011.
\end{innerlist}

\section{Supervision of bachelor students}
\vspace{-1em}
\begin{innerlist}
    \item Stefano Furi: \href{http://amslaurea.unibo.it/30280/}{\textit{Accesso e Controllo Efficiente di Sistemi Software Complessi tramite GraphQL}}, 2023.
    \item Kelvin Oluwada Milare Obuneme Olaiya: \href{http://amslaurea.unibo.it/26455/}{\textit{Simulazione event-driven di folle con micro-interazione fisica ed elementi cognitivi}}, 2022.
    \item Andrea Acampora: \href{https://amslaurea.unibo.it/24704/}{\textit{Ri-progettazione del modulo di esportazione dati del simulatore Alchemist}}, 2021
    \item Simone Bonasegale: \href{https://amslaurea.unibo.it/23316/}{\textit{Build automation systems per Java: analisi dello stato dell'arte}}, 2021
    \item Andrea Zattoni: \href{https://archive.ph/CnUXg}{\textit{Una piattaforma Peer-to-Peer universale per Aggregate Computing}}, 2020.
    \item Vuksa Mihajlovic: \href{http://amslaurea.unibo.it/21648/}{\textit{Sviluppo di interfacce grafiche moderne in Kotlin e JavaFX: evoluzione della UI del simulatore Alchemist}}, 2020.
    \item Lorenzo Paganelli: \href{https://amslaurea.unibo.it/20540/}{\textit{Simulazione di evacuazione di folle in Alchemist: un modello di mappa mentale per pedoni cognitivi}}, 2020.
    \item Stefano Rasi: \href{https://amslaurea.unibo.it/20505/}{\textit{Manipolazione di Bytecode Java con la libreria ASM}}, 2020.
    \item Filippo Nardini: \href{https://amslaurea.unibo.it/19778/}{\textit{Sviluppo di piattaforme per il linguaggio Protelis in Kotlin e Java}}, 2019.
    \item Federico Pettinari: \href{https://amslaurea.unibo.it/19092/}{\textit{Un Framework per Simulazione e Sviluppo di Sistemi Aggregati di Smart-Camera}}, 2019.
    \item Diego Mazzieri: \href{https://amslaurea.unibo.it/19084/}{\textit{Progettazione e implementazione di agenti cognitivi per simulazioni di evacuazioni di folle in Alchemist}}, 2019.
    \item Luca Giuliani: \href{https://amslaurea.unibo.it/19071/}{\textit{Progettazione e Implementazione di un Domain Specific Language per la Costruzione di Reti Geniche}}, 2019.
    \item Manuele Pasini: \href{http://amslaurea.unibo.it/18535/}{\textit{Programmazione memory-safe senza garbage collection: il caso del linguaggio Rust}}, 2019.
    \item Nicolas Barilari: \href{http://amslaurea.unibo.it/16841/}{\textit{Programmazione Reattiva in Kotlin su sistemi Android}}, 2018.
    \item Luca Casamenti: \href{http://amslaurea.unibo.it/16788/}{\textit{Il linguaggio Ceylon}}, 2018.
    \item Davide Bondi: \href{http://amslaurea.unibo.it/15730/}{\textit{Protocollo LoRaWAN e IoT: interfacciamento con Java e sperimentazione su comunicazioni indoor}}, 2018.
    \item Matteo Magnani: \href{http://amslaurea.unibo.it/17133/}{\textit{Design e implementazione di un sistema di grid computing per il simulatore Alchemist}}, 2017.
    \item Niccolò Maltoni: \href{http://amslaurea.unibo.it/14682/}{\textit{Progettazione object-oriented di un'interfaccia grafica JavaFX per il simulatore Alchemist}}, 2017.
    \item Luca Semprini: \href{http://amslaurea.unibo.it/14673/}{\textit{Una panoramica su Kotlin: il nuovo linguaggio per lo sviluppo di applicazioni Android}}, 2017.
    \item Andrea Placuzzi: \href{http://amslaurea.unibo.it/14329/}{\textit{Integrazione dei formati di navigazione GPS standard in Alchemist}}, 2017.
    \item Giacomo Scaparrotti: \href{http://amslaurea.unibo.it/14019/}{\textit{Studio delle prestazioni del simulatore Alchemist: ottimizzazione di routing e caching}}, 2017.
    \item Elisa Casadio: \href{http://amslaurea.unibo.it/12310/}{\textit{Revisione e refactoring dell'interfaccia utente del simulatore Alchemist}}, 2016.
    \item Gianluca Grossi: \href{http://amslaurea.unibo.it/12503/}{\textit{Sviluppo di plugin per IntelliJ IDEA}}, 2016.
    \item Giovanni Romio: \href{http://amslaurea.unibo.it/10481/}{\textit{Backport di una applicazione da Java 8 a Java 7}}, 2016.
    \item Francesco Cardi: \href{http://archive.fo/zMGo8}{\textit{Sapere Adaptive Visualisation}}, 2012.
\end{innerlist}

\section{Extra-institutional teaching}
\href{https://www.bbs.unibo.eu/hp/}{\textbf{Bologna Business School}}, Bologna (BO), Italy
\begin{outerlist}
\item[] \textbf{Master Courses} %\hfill \textbf{October 2018 to December 2018}
    \begin{innerlist}
        \item \textit{Internet of Things -- Software production} --- advanced course on techniques for producing high quality software for the IoT. Focus on team coordination strategies and tools, build automation, testing, continuous integration, and continuous delivery, 4 hours.
\end{innerlist}
\halfblankline
\end{outerlist}

\href{http://www.formart.it/}{\textbf{FORMart}}, Cesena (FC), Italy
\begin{outerlist}
\item[] \textbf{Istruzione e Formazione Tecnica Superiore}
    \begin{innerlist}
        \item \textit{Internet of Things} --- Introduction to distributed computing and to the Internet of Things, with focus on Industry 4.0, 2019, 30 hours
        \item \textit{Programmazione e ICT problem solving} --- course on algorithmic problem resolution and automation, with elements of programming in Python, 2018, 30 hours
        \item \textit{Sistemi informatici e loro gestione} --- course on basics of operating systems, networking, and database management, 2017, 30 hours
        \item \textit{Elementi di Programmazione e Sviluppo di Applicazioni} --- course on imperative and object oriented programming with C and Java, 2016, 30 hours
    \end{innerlist}
\halfblankline
\end{outerlist}


\newpage
\macrosection{OTHER ACTIVITIES AND SKILLS}

\halfblankline{}

\colorlet{languagecolor}{blue}
\colorlet{nolanguagecolor}{gray}
\newcount\languagecount
\newcommand\languageknowledge[2]
  {%
  \hbox
      {%
%         \makebox[10cm][l]{#1}%
        \languagecount=0
        \loop\ifnum\languagecount<#2
          \advance\languagecount1
          \textcolor{languagecolor}{$\bullet$}%
        \repeat
        \loop\ifnum\languagecount<10
          \advance\languagecount1
          \textcolor{nolanguagecolor}{$\bullet$}%
        \repeat
        \hspace{5pt} #1
      }%

    }

\begin{minipage}{\textwidth}
\textbf{Note:} the skill self-evaluation scale has the following meaning:

\languageknowledge{No experience}{0}
\languageknowledge{Can read programs, make small changes to existing programs}{1}
\languageknowledge{Can write hello world without looking at a book}{2}
\languageknowledge{Can utilize basic features without much help}{3}
\languageknowledge{Can develop medium programs and to do nontrivial troubleshooting}{4}
\languageknowledge{Can develop large programs using all basic and most esoteric features}{5}
\languageknowledge{Can develop large programs and deploy new systems from scratch}{6}
\languageknowledge{Understanding and (appropriate) usage of most lesser known features}{7}
\languageknowledge{Deep understanding of corner cases and esoteric features}{8}
\languageknowledge{Could have written the book, but didn't}{9}
\languageknowledge{Wrote the book or paper (there must be a book or paper)}{10}
\end{minipage}


\section{Professional experience}
\href{http://www.peer-network.it/}{\textbf{Peer Network}}, Ravenna (RA), Italy
\begin{outerlist}
\item[] \textit{Consulting on DevOps, version control, build automation, continuous integration, and Kotlin programming}%
        \hfill \textbf{2021}
\end{outerlist}
\halfblankline

\href{http://www.twinlogix.com/en}{\textbf{twinlogix}}, Santarcangelo di Romagna (RN), Italy
\begin{outerlist}
\item[] \textit{Consulting on code generators}%
        \hfill \textbf{2016, 2021}
\end{outerlist}
\halfblankline

\href{http://www.valpharma.com/}{\textbf{Valpharma International S.p.A.}}, Pennabilli (RN), Italy
\begin{outerlist}
\item[] \textit{Stage: raw material quality control}%
        \hfill \textbf{July 2004 to August 2004}
\end{outerlist}
\halfblankline


\section{Proficiency with programming languages}
\vspace{-20pt}
\begin{center}
\begin{tabular}{l  l  l}
\languageknowledge{Bash}{3}
&
\languageknowledge{C}{4}
&
\languageknowledge{C$++$}{3}
\\
\languageknowledge{C\#}{3}
&
\languageknowledge{Groovy}{4}
&
\languageknowledge{Kotlin}{6}
\\
\languageknowledge{Java}{8}
&
\languageknowledge{JavaScript}{2}
&
\languageknowledge{Prolog}{3}
\\
\languageknowledge{Protelis}{10}
&
\languageknowledge{Python}{4}
&
\languageknowledge{Ruby}{3}
\\
\languageknowledge{Rust}{1}
&
\languageknowledge{Scala}{4}
&
\languageknowledge{TypeScript}{2}
\\
\languageknowledge{Xtend}{5}
&
&
\end{tabular}
\end{center}

\vspace{5pt}

\section{Proficiency with other languages}
\vspace{-20pt}
\begin{center}
\begin{tabular}{l l l}
\languageknowledge{CSS}{3}
&
\languageknowledge{HTML}{4}
&
\languageknowledge{JSON}{8}
\\
\languageknowledge{\LaTeX}{3}
&
\languageknowledge{Markdown}{7}
&
\languageknowledge{SQL}{3}
\\
\languageknowledge{XML}{6}
&
\languageknowledge{XSLT}{3}
&
\languageknowledge{YAML}{7}
\end{tabular}
\end{center}

\vspace{5pt}

\section{Proficiency with software tools}
\vspace{-20pt}
\begin{center}
\begin{tabular}{l l}
\languageknowledge{Apache Maven}{6}
&
\languageknowledge{Blender}{2}
\\
\languageknowledge{GIMP}{3}
&
\languageknowledge{Git}{7}
\\
\languageknowledge{GitHub Actions}{7}
&
\languageknowledge{Gradle}{7}
\\
\languageknowledge{Hugo}{4}
&
\languageknowledge{Inkscape}{4}
\\
\languageknowledge{Jekyll-rb}{4}
&
\languageknowledge{Kdenlive}{4}
\\
\languageknowledge{Mercurial}{4}
&
\languageknowledge{Spreadsheets}{5}
\\
\languageknowledge{Subversion}{3}
&
\languageknowledge{Travis CI}{7}
\\
\languageknowledge{Xtext}{6}
&
\end{tabular}
\end{center}

\vspace{5pt}

\section{Proficiency with operating systems}
\vspace{-30pt}
\begin{center}
\begin{tabular}{l l}
\languageknowledge{Arch Linux and derivatives}{7}
&
\languageknowledge{Gentoo Linux and derivatives}{6}
\\
\languageknowledge{Mac OS X}{4}
&
\languageknowledge{Microsoft Windows}{4}
\end{tabular}
\end{center}

\section{Lead designer of software projects}
Lead designer and major developer of \href{http://protelis.org/}{Protelis}, 2014--today
\begin{innerlist}
    \item Protelis is a programming language aiming at making networked systems just as easy to build for complex and heterogeneous networks as for single machines and cloud systems. This accomplished by separating the different tasks and making some of the hard and subtle parts automatic and implicit.
\end{innerlist}
\halfblankline

Lead designer and major developer of \href{http://alchemistsimulator.github.io/}{Alchemist}, 2010--today
\begin{innerlist}
    \item Alchemist is an innovative simulator meant to join the expressiveness of the agent based modelling and the power and speed of the stochastic simulation algorithms used in chemistry. It is tailored to scenarios in which many nodes interact exchanging  informations. Its flexibility allows for a wide range of applications, spacing from the classical chemistry to the biology (e.g. complex morphogenesis processes) to pervasive computing.
\end{innerlist}
\halfblankline

Lead designer and developer of \href{https://github.com/DanySK/refreshversions-aliases}{refreshversions-aliases}, 2020--2022
\begin{innerlist}
    \item A set of version grouping rules applicable to the well known refreshVersions Gradle plugin,
    with the goal fo grouping and synchronizing version upgrades.
\end{innerlist}
\halfblankline

Lead designer and developer of \href{https://github.com/DanySK/gradle-latex}{Gradle LaTeX plugin}, 2019--today
\begin{innerlist}
    \item A Gradle plugin for compiling LaTeX.
\end{innerlist}
\halfblankline

Lead designer and developer of \href{https://github.com/DanySK/git-sensitive-semantic-versioning-gradle-plugin}{Git sensitive Semantic Versioning (SemVer) Gradle Plugin}, 2019--today
\begin{innerlist}
    \item A Gradle plugin that applies Semantic Versioning to projects based on the status of the git repository.
\end{innerlist}
\halfblankline

Lead designer and developer to \href{https://github.com/DanySK/publish-on-central}{publish-on-central}, 2019--today
\begin{innerlist}
    \item A Gradle plugin for streamlined publishing on Maven Central
\end{innerlist}
\halfblankline

Lead designer and developer of \href{https://github.com/DanySK/jirf}{JIRF}, 2017--today
\begin{innerlist}
    \item The Java Implicit Reflective Factory allows for building objects reflectively inside configured contexts, applying implicit type conversions chains when needed.
\end{innerlist}
\halfblankline

Lead designer and leading developer of \href{https://github.com/DanySK/javadoc.io-linker}{Javadoc.io Linker}, 2016--today
\begin{innerlist}
    \item Javadoc.io linker is a Gradle plugin that configures any Javadoc build to link javadoc.io when referring to non-local classes.
\end{innerlist}
\halfblankline

Lead designer and developer of \href{https://github.com/DanySK/refreshversions-aliases}{refreshversions-aliases}, 2020--2022
\begin{innerlist}
    \item A set of version grouping rules applicable to the well known refreshVersions Gradle plugin,
    with the goal fo grouping and synchronizing version upgrades.
\end{innerlist}
\halfblankline

Lead designer and developer of \href{https://github.com/DanySK/upgradle}{UpGradle}, 2020--2022
\begin{innerlist}
    \item A confgurable and extendable bot that automatically searches and applies software updates,
        opening pull requests on GitHub. Focused on Gradle.
\end{innerlist}
\halfblankline

Lead designer and leading developer of \href{https://github.com/DanySK/urlclassloader-util}{URLClassloader Util}, 2016--2021
\begin{innerlist}
    \item URLClassloader Util is a small library that provides functionality to manipulate the Java classpath at runtime.
\end{innerlist}
\halfblankline

Lead designer and developer of \href{https://github.com/DanySK/SmarTrRR}{SmarTrRR}, 2015--2017
\begin{innerlist}
    \item SmarTrRR is a transitive dependency range resolver plugin for Gradle. It replaces the default Protelis resolver, implementing a progressive range restriction, and a conflict resolution algorithm. Also, it allows the user to configure specific artifact substitutions.
\end{innerlist}
\halfblankline

Lead designer and developer of \href{https://sourceforge.net/projects/mandelbrot/}{Angela the Mandelbrot Set Explorer}, 2009
\begin{innerlist}
  \item Angela is a Java parallel application that allows for visualizing portions of the Mandelbrot set.
\end{innerlist}
\halfblankline

\section{Contribution to open source software}
Contributor to \href{https://github.com/renovatebot/renovate}{Mend Renovate}, 2020--today
\begin{innerlist}
    \item Renovate is a bot that checks for software dependency updates automatically.
    I contributed by writing support for the Gradle Dependency Catalogs.
\end{innerlist}
\halfblankline

Contributor to \href{https://github.com/DanySK/khttp}{khttp}, 2020--today
\begin{innerlist}
    \item HTTP in Kotlin made easy. Repackaging and publication on Maven Central.
\end{innerlist}
\halfblankline

Contributor to \href{https://github.com/DanySK/Thread-Inheritable-Resource-Loader-for-Java}{Thread-inheritable resource loader for Java}, 2017--today
\begin{innerlist}
    \item A statically-usable resource and class loader that inherits the parent thread's class loader.
\end{innerlist}
\halfblankline

Contributor to \href{https://github.com/DanySK/gson-extras}{gson-extras}, 2017--today
\begin{innerlist}
    \item Extra libraries and component for Google Gson, extracted from the main repository and made publicly available on Maven Central.
\end{innerlist}
\halfblankline

Contributor to \href{https://github.com/edvin/tornadofx}{TornadoFX}, porting from JDK8 to JDK11+, 2019
\begin{innerlist}
    \item TornadoFX is a Kotlin DSL for building rich graphical applications with JavaFX.
\end{innerlist}
\halfblankline

Contributor to \href{https://github.com/farkam135/GoIV}{GoIV}, 2017
\begin{innerlist}
    \item GoIV is an Android application devoted to rating the quality of Pokémon Go monsters relying solely on the on-screen information.
\end{innerlist}
\halfblankline

Contributor to \href{https://github.com/Antergos/Cnchi}{Cnchi}, 2015
\begin{innerlist}
    \item Cnchi is a modern, flexible installer for Linux, developed by the Antergos Linux team.
\end{innerlist}
\halfblankline

Creator and maintainer of the following Arch User Repository Packages, 2018--2019
\begin{innerlist}
    \item \href{https://aur.archlinux.org/packages/opencorsairlink-git/}{\texttt{opencorsairlink-git}}.
    \item \href{https://aur.archlinux.org/packages/opencorsairlink-testing-git/}{\texttt{opencorsairlink-testing-git}}.
\end{innerlist}
\halfblankline

Creator and maintainer of the \href{https://bitbucket.org/danysk/nirvana-overlay/}{Nirvana overlay} for Gentoo Linux, 2014--2015
\begin{innerlist}
    \item Nirvana is an overlay for Gentoo Linux, namely a container of ebuild files, which are scripts describing how to install and maintain packages in a Gentoo Linux distribution. Nirvana contains those ebuild that work well, but are too hard to maintain to be pushed in Sunrise or Sabayon overlays. Moreover, this repository is used by me as a playground for creating new ebuilds. On July 2014 Nirvana got officially indexed by Layman, and as a consequence it is now available to all Gentoo users using such tool.
\end{innerlist}
\halfblankline

Creator and maintainer of the {Nirvana Community Repository}, 2014--2015
\begin{innerlist}
    \item Nirvana Community Repository contains the same packages included in Nirvana overlay, distributed in a pre-compiled form compatible with Sabayon Linux Entropy package manager.
\end{innerlist}
\halfblankline

Member of the testing and development teams of \href{http://www.sabayon.org/}{Sabayon Linux}, 2008--2014
\begin{innerlist}
  \item Sabayon Linux is a Gentoo-based distribution which follows the works-out-of-the-box philosophy, aiming to give the user a wide number of applications that are ready for use and a self-configured operating system.
\end{innerlist}
\halfblankline

\section{Service in non-profit organizations}
Frequent contributor to \href{http://www.wikipedia.org/}{Wikipedia} and \href{http://www.openstreetmap.org/}{OpenStreetMap}.

\halfblankline

\href{http://www.astice.org/}{A.St.I.Ce. Executive Board Member}, January 2006 to November 2009
    \begin{innerlist}
      \item Founded ``I$^2$ --- Informa Ingegneri'', the technical journal of Seconda Facoltà di Ingegneria, containing articles about the research activity of the faculty.
      \item Founded ``Linux Libera Tutti'', a project meant to allow students access without any charge DVDs and CDs of various Linux distributions, with a special focus on Sabayon Linux.
    \end{innerlist}
\halfblankline

% \section{Expertise}
% Mathematics:
% \begin{innerlist}
%     \item Applied Mathematics, Real and Complex Analysis, Discrete Mathematics, Geometry.
% \end{innerlist}
% \halfblankline
%
% Physics:
% \begin{innerlist}
%     \item Mechanics, Electromagnetism.
% \end{innerlist}
% \halfblankline
%
% Control Theory and Engineering:
% \begin{innerlist}
%     \item Distributed and Self-adaptive Control, Dynamic Optimization, Bio-mimicry, Bio-inspiration.
% \end{innerlist}
% \halfblankline
%
% Communications and Signal Processing:
% \begin{innerlist}
%     \item Probability, Random Variables, Stochastic Processes, Networks
% \end{innerlist}
% \halfblankline
%
% Computer Science and Engineering:
% \begin{innerlist}
%     \item Model Checking, Software Verification, Component-Based Reusable Software, Object Oriented Programming, Logic Programming, Functional Programming, Concurrent Programming, Distributed Systems, Benchmarking, Model Driven Software Development.
% \end{innerlist}
% \halfblankline
%
% Natural Sciences (Biology, Microbiology, Chemistry, Biochemistry):
% \begin{innerlist}
%     \item Molecular orbital theory, stoichiometry, organic chemistry, DNA transcription and replication processes, PCR, metabolic processes, virus classification, bacteria classification, human morphology, physiology, Earth sciences, astronomy.
% \end{innerlist}



\let\thefootnote\relax\footnotetext{\today}

\end{document}

%%%%%%%%%%%%%%%%%%%%%%%%%% End CV Document %%%%%%%%%%%%%%%%%%%%%%%%%%%%%

%----------------------------------------------------------------------%
% The following is copyright and licensing information for
% redistribution of this LaTeX source code; it also includes a liability
% statement. If this source code is not being redistributed to others,
% it may be omitted. It has no effect on the function of the above code.
%----------------------------------------------------------------------%
% Copyright (c) 2007, 2008, 2009, 2010, 2011 by Theodore P. Pavlic
%
% Unless otherwise expressly stated, this work is licensed under the
% Creative Commons Attribution-Noncommercial 3.0 United States License. To
% view a copy of this license, visit
% http://creativecommons.org/licenses/by-nc/3.0/us/ or send a letter to
% Creative Commons, 171 Second Street, Suite 300, San Francisco,
% California, 94105, USA.
%
% THE SOFTWARE IS PROVIDED "AS IS", WITHOUT WARRANTY OF ANY KIND, EXPRESS
% OR IMPLIED, INCLUDING BUT NOT LIMITED TO THE WARRANTIES OF
% MERCHANTABILITY, FITNESS FOR A PARTICULAR PURPOSE AND NONINFRINGEMENT.
% IN NO EVENT SHALL THE AUTHORS OR COPYRIGHT HOLDERS BE LIABLE FOR ANY
% CLAIM, DAMAGES OR OTHER LIABILITY, WHETHER IN AN ACTION OF CONTRACT,
% TORT OR OTHERWISE, ARISING FROM, OUT OF OR IN CONNECTION WITH THE
% SOFTWARE OR THE USE OR OTHER DEALINGS IN THE SOFTWARE.
%----------------------------------------------------------------------%
