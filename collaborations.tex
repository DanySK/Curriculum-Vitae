% ! TeX root = curriculum.tex
\textbf{APICe Research group} \hfill \textbf{March 2011 --- today}\\
The primary research group I worked with since the conclusion of my two years master. The group includes professors Andrea Omicini, Mirko Viroli, Antonio Natali, Alessandro Ricci, and operates at the University of Bologna.

\halfblankline{}

\textbf{DISI/Raytheon BBN Technologies} \hfill \textbf{November 2013 --- today}\\
Collaboration between my research unit and the research team in Raytheon BBN Technologies led by dr. Jacob Beal. The collaboration fostered the birth of aggregate computing. A decisive step forward was taken in 2014, when I visited for three months, designing and developing the Protelis programming language. I visited again dr. Jacob Beal at the University of Iowa in 2016. The collaboration recently led to a research contract of \$40.000 between BBN and UniBo.

\halfblankline{}

\textbf{CommonWears consortium} \hfill \textbf{September 2022 --- today}\\
Consortium of five italian universities, PRIN 2022 Italian project.
\begin{innerlist}
    \item Università della Calabria
    \item Alma Mater Studiorum---Università di Bologna
    \item Free University of Bozen-Bolzano
    \item Università Campus Bio-Medico di Roma
    \item Università di Torino
\end{innerlist}
Among others, the consortium includes professors Giancarlo Fortino, Ferruccio Damiani, and Antonio Liotta.

\halfblankline{}

\textbf{FluidWare consortium} \hfill \textbf{September 2019 --- today}\\
Consortium of four italian universities, PRIN 2017 Italian project.
\begin{innerlist}
    \item Università degli studi di Modena e Reggio Emilia
    \item Alma Mater Studiorum---Università di Bologna
    \item Università di Camerino
    \item Università della Calabria
\end{innerlist}
Among others, the consortium includes professors Franco Zambonelli, Giancarlo Fortino and Barbara Re.

\halfblankline{}

\textbf{Aggregate computing research group} \hfill \textbf{2015 --- today}\\
International collaboration between:
\begin{innerlist}
    \item Alma Mater Studiorum---Università di Bologna;
    \item Raytheon BBN Technologies; and
    \item Università di Torino.
\end{innerlist}
The research group is envisioning, developing, and testing methods and technologies for conceiving systems with a focus on their global behavior. The UniBo team is composed by a subset of the APICe Research group, while the two other teams are led dr. Jacob Beal and prof. Ferruccio Damiani.

\halfblankline{}

\textbf{SAPERE Consortium} \hfill \textbf{March 2011 --- September 2013}\\
International consortium of five universities, within which I worked during the first half of my PhD.
\begin{innerlist}
    \item Università degli studi di Modena e Reggio Emilia
    \item Alma Mater Studiorum---Università di Bologna
    \item Johannes Kepler Universität Linz
    \item University of St Andrews
    \item Université de Genève
\end{innerlist}
The participation allowed me to work with professors Franco Zambonelli, Marco Mamei, Alois Ferscha, Simon Dobson, and Giovanna Di Marzo Serugendo, and their research groups.

\halfblankline{}

\textbf{Institut für Pervasive Computing} \hfill \textbf{July 2013 --- October 2013}\\
Research group coordinated by prof. Alois Ferscha. I had been working with the team for three months, focussing on analysis of pedestrian data to prevent crowd disasters.

\halfblankline{}

\textbf{Distributed systems @ Florida Tech} \hfill \textbf{July 2009 --- October 2009}\\
My first research activity has been carried on with prof. Ronaldo Menezes at the Florida Institute of Technologies while I was still a master student. I joined his research team for three months, studying the coordination of distributed systems and complex networks.

\halfblankline{}
