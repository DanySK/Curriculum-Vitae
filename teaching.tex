% ! TeX root = curriculum.tex

{
\scriptsize{
\textbf{NOTE:} In most courses in Italy students are mandatorily submitted an anonymous form where they can express their opinion about the course in a four-valued scale (very negative, negative, positive, very positive). Such evaluations are reported here when available, in the form $(A, B, C, D, E)$, where:
\\$A$: Overall course satisfaction;
\\$B$: Availability of the teacher;
\\$C$: Clarity of exposition;
\\$D$: The teacher stimulates learning interest;
\\$E$: Number of respondents.
\\With the exception of $E$, the fraction of students evaluating positively or very positively is reported.
}
}

\newcommand{\shortcourse}[6]{
    \href{#1}{\textit{\textbf{#2}}},
    \href{#3}{#4},
    \href{#5}{#6}}

\newcommand{\course}[9]{
    \shortcourse{#1}{#2}{#3}{#4}{#5}{#6},
    #7,
    #8.
    \textit{#9}
}

\newcommand{\shortunibocourse}[4]{
    \shortcourse{#1}{#2}
    {http://www.unibo.it}{Alma Mater Studiorum---Università di Bologna}
    {#3}{#4}}

\newcommand{\unibocourse}[7]{
    \course{#1}{#2}
    {http://www.unibo.it}{Alma Mater Studiorum---Università di Bologna}
    {#3}{#4}
    {#5}
    {#6}{#7}
}

\section{PhD courses}
\vspace{-2em}
\begin{outerlist}
    \item[2021/22]
        \unibocourse
        {LINK-TO-SLIDES-TO-BE-PREPARED}
        {DevOps meets scientific research}
        {https://archive.ph/pH7Hk}
        {PhD program in Computer Science and Engineering }
        {T.B.D.}
        {20 hours}
        {Application of DevOps methodologies to scientific artifacts, reproducibility, sharing, licensing, versioning.}
    \item[2019/20]
        \unibocourse
        {https://bitbucket.org/danysk/courses-2018-developing-maintaining-and-sharing-software-tools/downloads/}
        {Developing, Maintaining, and Sharing Software Tools for Research}
        {http://archive.fo/Fmxxm}
        {PhD program in Data Science and Computation}
        {October 2019}
        {8 hours}
        {Creation of shareable and reproducible research software through DevOps methodologies; licensing.}
    \item[2019/20]
        \unibocourse
        {http://archive.fo/HKAC6/}
        {DevOps for scientific research}
        {http://archive.fo/ppTiB}
        {Phd Program in Computer Science and Engineering}
        {September 2019}
        {20 hours}
        {Creation of shareable and reproducible research software through DevOps tools and methodologies; versioning; licensing.}
    \item[2018/19]
        \unibocourse
        {https://bitbucket.org/danysk/courses-2018-developing-maintaining-and-sharing-software-tools/downloads/}
        {Developing, Maintaining, and Sharing Software Tools for Research}
        {http://archive.fo/Fmxxm}
        {PhD program in Data Science and Computation}
        {2018}
        {20 hours}
        {Creation of shareable and reproducible research software through DevOps tools and methodologies; versioning; licensing.}
\end{outerlist}

\section{Course Responsibility}
\vspace{-2em}
\begin{outerlist}
    \item[2021/22]
        \unibocourse
        {https://archive.ph/75yeO}{Laboratory of Software Systems (Laboratorio di Sistemi Software---Titolare del corso)}
        {https://archive.ph/ljEQn}{Two year master in Computer Science and Engineering (Laurea magistrale in Ingegneria e Scienze informatiche)}
        {3 CFU/ECTS, 30 hours, \textasciitilde{}30 students}
        {(76.5\%, 88.2\%, 76.5\%, 88.2\%, 17)}
        {Internal DSLs in Kotlin and their use in build systems, DevOps practices, build automation (Gradle as paradigmatic automator), continuous integration, advanced version control and team organization, licensing, open source sharing.}
    \item[2020/21]
        \unibocourse
        {https://archive.is/CyIjt}{Laboratory of Software Systems (Laboratorio di Sistemi Software---Titolare del corso)}
        {https://archive.is/Svmye}{Two year master in Computer Science and Engineering (Laurea magistrale in Ingegneria e Scienze informatiche)}
        {3 CFU/ECTS, 30 hours, \textasciitilde{}30 students}
        {(96\%, 96\%, 95.8\%, 100\%, 25)}
        {Internal DSLs in Kotlin and their use in build systems, DevOps practices, build automation (Gradle as paradigmatic automator), continuous integration, advanced version control and team organization, licensing, open source sharing.}
\end{outerlist}

\section{Teaching of learning modules}
\vspace{-1.9em}
\newcommand{\oopjava}{Practical OOP programming in Java from simple examples to advanced mechanisms and design patterns, including parallel and functional-like programming; basics of quality assurance, profiling, and build automation; distributed version control.}
\newcommand{\sedtmeit}{DevOps: Principles, practices, and tools; agile development and SCRUM; testing; continuous integration and delivery.}
\begin{outerlist}
    \item[2021/22]
        \unibocourse %DTM
        {https://archive.ph/OxWCT}{Software Engineering}
        {https://archive.ph/fs8a6}{Two year master in Digital Transformation Management}
        {2 CFU/ECTS, 18 hours}
        {(90.9\%, 100\%, 100\%, 100\%, 22)}
        {
            \sedtmeit{}
            Note: this course module is shared with ``Computer Science and Engineering''.
        }
    \item[2021/22]
        \unibocourse %EIT
        {https://archive.ph/1mBPz}{Software Engineering for Intelligent Distributed Systems}
        {https://archive.ph/ljEQn}{Two year master in Computer Science and Engineering}
        {2 CFU/ECTS, 18 hours}
        {(90.9\%, 100\%, 100\%, 100\%, 22)}
        {
            \sedtmeit{}
            Note: this course module is shared with ``Digital Transformation Management''.
        }
    \item[2021/22]
        \unibocourse
        {https://archive.ph/oUCdx}{Object-Oriented Programming (Programmazione ad Oggetti)}
        {https://archive.ph/vKgXj}{Bachelor in Computer Science and Engineering (Laurea in Ingegneria e scienze informatiche)}
        {3 CFU/ECTS, 30 hours, \textasciitilde{}150 students}
        {(91.1\%, 99.1\%, 93.6\%, 92.8\%, 112)}
        {\oopjava}
    \item[2020/21]
        \unibocourse
        {https://archive.is/WxwNn}{Object-Oriented Programming (Programmazione ad Oggetti)}
        {https://archive.is/WeNI1}{Bachelor in Computer Science and Engineering (Laurea in Ingegneria e scienze informatiche)}
        {3 CFU/ECTS, 30 hours, \textasciitilde{}150 students}
        {(77.7\%, 96.4\%, 79.5\%, 79.5\%, 112)}
        {\oopjava}
    \item[2019/20]
        \unibocourse
        {http://archive.fo/JtEDW}{Object-Oriented Programming (Programmazione ad Oggetti)}
        {http://archive.fo/UM5wl}{Bachelor in Computer Science and Engineering (Laurea in Ingegneria e scienze informatiche)}
        {3 CFU/ECTS, 30 hours, \textasciitilde{}150 students}
        {(83.6\%, 99.2\%, 86.0\%, 90.8\%, 130)}
        {\oopjava}
    \item[2018/19]
        \unibocourse
        {http://archive.fo/srdtN}{Object-Oriented Programming (Programmazione ad Oggetti)}
        {http://archive.fo/UM5wl}{Bachelor in Computer Science and Engineering (Laurea in Ingegneria e scienze informatiche)}
        {3 CFU/ECTS, 30 hours, \textasciitilde{}150 students}
        {(91.3\%, 99.0\%, 91.3\%, 91.3\%, 104)}
        {\oopjava}
    \item[2017/18]
        \unibocourse
        {http://archive.fo/54lT9}{Object-Oriented Programming (Programmazione ad Oggetti)}
        {http://archive.fo/UM5wl}{Bachelor in Computer Science and Engineering (Laurea in Ingegneria e scienze informatiche)}
        {3 CFU/ECTS, 30 hours, \textasciitilde{}150 students}
        {(87.9\%, 99.1\%, 83.5\%, 94.0\%, 116)}
        {\oopjava}
    \item[2016/17]
        \unibocourse
        {http://archive.fo/0XLbd}{Object-Oriented Programming (Programmazione ad Oggetti)}
        {http://archive.fo/UM5wl}{Bachelor in Computer Science and Engineering (Laurea in Ingegneria e scienze informatiche)}
        {3 CFU/ECTS, 30 hours, \textasciitilde{}150 students}
        {(82.1\%, 97.5\%, 95.1\%, 90.2\%, 123)}
        {\oopjava}
    \item[2015/16]
        \unibocourse
        {http://archive.fo/1eIMt}{
            Engineering Complex Adaptive Software Systems
            (Ingegneria dei Sistemi Software Adattativi Complessi)
        }
        {http://archive.fo/toz5c}{Two-year master in Computer Science and Engineering (Laurea magistrale in Ingegneria e Scienze informatiche)}
        {1 CFU/ECTS, 9 hours, \textasciitilde{}10 students}
        {(100\%, 100\%, 100\%, 100\%, 6)}
        {Distributed self-organization using programmable tuple spaces; simulation and reproducibility.}
    \item[2014/15]
        \unibocourse
        {http://archive.fo/JrWEu/}{Foundations of Informatics A (Fondamenti di Informatica A)}
        {http://archive.fo/30rN0}{Bachelor in Electronics Engineering for Energy and Information (Laurea in Ingegneria elettronica)}
        {3 CFU/ECTS, 30 hours, \textasciitilde{}75 students}
        {(88.7\%, 95.1\%, 98.4\%, 98.4\%, 62)}
        {
            Practical imperative programming in C, from foundations to graphical interfaces.
            Note: this course module is shared with ``Bachelor in Biomedical Engineering''.
        }
    \item[2014/15]
        \unibocourse
        {http://archive.fo/JrWEu/}{Foundations of Informatics A (Fondamenti di Informatica A)}
        {http://archive.fo/jW52L}{Bachelor in Biomedical Engineering (Laurea in Ingegneria biomedica)}
        {3 CFU/ECTS, 30 hours, \textasciitilde{}75 students}
        {(88.7\%, 95.1\%, 98.4\%, 98.4\%, 62)}
        {
            Practical imperative programming in C, from foundations to graphical interfaces.
            Note: this course module is shared with ``Bachelor in Electronics Engineering for Energy and Information''.
        }
    \item[2011/12]
        \unibocourse
        {http://archive.fo/nnsBl/}
        {Multi-Agent Systems (Sistemi multi-agente LM)}
        {http://archive.fo/qDVq3}{Two-year master in Computer Engineering (Laurea magistrale in Ingegneria Informatica)}
        {3 CFU/ECTS, 30 hours, \textasciitilde{}15 students}
        {(90.9\%, 100\%, 100\%, 100\%, 11)}
        {Practical multi-agent design and programming; distributed multi-agent platforms (Jade), innovative paradigms and programming languages (Jason).}
\end{outerlist}

\section{Tutoring}
\vspace{-1.9em}
\begin{outerlist}
    \item[2019/20]
        \shortunibocourse
        {http://archive.fo/JtEDW}{Object-Oriented Programming}
        {http://archive.fo/UM5wl}{Bachelor in Computer Science and Engineering}, \textasciitilde{}150 students.
    \item[2018/19]
        \shortunibocourse
        {http://archive.fo/srdtN}{Object-Oriented Programming}
        {http://archive.fo/UM5wl}{Bachelor in Computer Science and Engineering}, \textasciitilde{}150 students.
    \item[2015/16]
        \shortunibocourse
        {http://archive.fo/puTDG}{Object-Oriented Programming}
        {http://archive.fo/UM5wl}{Bachelor in Computer Science and Engineering}, \textasciitilde{}150 students.
    \item[2014/15]
        \shortunibocourse
        {http://archive.fo/5LhhW}{Engineering Complex Adaptive Software Systems}
        {http://archive.fo/toz5c}{Two-year master in Computer Science and Engineering}.
    \item[2014/15]
        \shortunibocourse
        {http://archive.fo/8jzEp}{Object-Oriented Programming}
        {http://archive.fo/UM5wl}{Bachelor in Computer Science and Engineering}, \textasciitilde{}150 students.
    \item[2013/14]
        \shortunibocourse
        {http://archive.fo/h8JCD}{Engineering Complex Adaptive Software Systems}
        {http://archive.fo/toz5c}{Two-year master in Computer Science and Engineering}.
    \item[2013/14]
        \shortunibocourse
        {http://archive.fo/0Gr16}{Object-Oriented Programming}
        {http://archive.fo/UM5wl}{Bachelor in Computer Science and Engineering}, \textasciitilde{}150 students.
    \item[2013/14]
        \shortunibocourse
        {http://archive.fo/XZFR0}{Foundations of Informatics A}
        {http://archive.fo/UM5wl}{Bachelor in Electronics Engineering for Energy and Information, Bachelor in Biomedical Engineering}, \textasciitilde{}75 students.
\end{outerlist}

\section{Supervision of graduate students}
\vspace{-1em}
\begin{innerlist}
    \item Kelvin Oluwada Milare Obuneme Olaiya: \href{http://amslaurea.unibo.it/26455/}{\textit{Simulazione event-driven di folle con micro-interazione fisica ed elementi cognitivi}}, 2022.
    \item Alessandro Neri: \href{http://amslaurea.unibo.it/23043/}{\textit{Da software monolitico a DevOps e Microservizi: un caso di studio industriale}}, 2021.
    \item Loris Cangini: \href{http://amslaurea.unibo.it/20410/}{\textit{Una piattaforma client-server universale per Aggregate Computing}}, 2020.
    \item Niccolò Maltoni: \href{http://amslaurea.unibo.it/20478/}{\textit{Progettazione di una piattaforma web per la simulazione di programmi aggregati}}, 2020.
    \item Andrea Placuzzi: \href{http://amslaurea.unibo.it/20484/}{\textit{A platform for aggregate computing over LoRaWAN network}}, 2020.
    \item Giacomo Scaparrotti: \href{http://amslaurea.unibo.it/20440/}{\textit{Cross-simulator integration: ns3 as a network simulation back-end for Alchemist}}, 2020.
    \item Filippo Nicolini: \href{http://amslaurea.unibo.it/19521/}{\textit{Simulazione di Agenti BDI basati su Prolog in Alchemist}}, 2019.
    \item Matteo Francia: \href{http://amslaurea.unibo.it/13090/}{\textit{A Foundational Library for Aggregate Programming}}, 2017.
    \item Simone Costanzi: \href{http://amslaurea.unibo.it/10519/}{\textit{Integrazione di piattaforme d'esecuzione e simulazione in una toolchain Scala per aggregate programming}}, 2016.
    \item Davide Ensini: \href{http://amslaurea.unibo.it/7990/}{\textit{Spatial computing per smart devices}}, 2014.
    \item Luca Nenni: \href{http://amslaurea.unibo.it/6927/}{\textit{Simulazioni realistiche di algoritmi di Crowd Steering}}, 2014.
    \item Enrico Polverelli: \href{http://amslaurea.unibo.it/5293/}{\textit{Simulazione di algoritmi di auto-organizzazione basati su gradiente computazionale in Alchemist}}, 2012.
    \item Andrea Dallatana: \href{http://amslaurea.unibo.it/4217/}{\textit{BDI agents for Real Time Strategy games}}, 2012.
    \item Francesca Cioffi: \href{http://amslaurea.unibo.it/4088/}{\textit{Algoritmi gradient-based per la modellazione e simulazione di sistemi auto-organizzanti}}, 2012.
    \item Paolo Contessi: \href{http://amslaurea.unibo.it/4074/}{\textit{Supporting semantic web technologies in the pervasive service ecosystems middleware}}, 2012.
    \item Giacomo Pronti: \href{http://archive.fo/nBeOg}{\textit{Simulazione di ecosistemi di servizi pervasivi con supporto ad annotazioni tuple based}}, 2012.
    \item Michele Morgagni: \href{http://archive.fo/6mnSN}{\textit{Modulo di comunicazione in una infrastruttura per pervasive service ecosystems}}, 2011.
    \item Matteo Desanti: \href{http://archive.fo/rwla1}{\textit{Supporto a regole chimico-semantiche per la coordinazione di service pervasive ecosystems}}, 2011.
\end{innerlist}

\section{Supervision of bachelor students}
\vspace{-1em}
\begin{innerlist}
    \item Andrea Acampora: \href{https://amslaurea.unibo.it/24704/}{\textit{Ri-progettazione del modulo di esportazione dati del simulatore Alchemist}}, 2021
    \item Simone Bonasegale: \href{https://amslaurea.unibo.it/23316/}{\textit{Build automation systems per Java: analisi dello stato dell'arte}}, 2021
    \item Andrea Zattoni: \href{https://archive.ph/CnUXg}{\textit{Una piattaforma Peer-to-Peer universale per Aggregate Computing}}, 2020.
    \item Vuksa Mihajlovic: \href{http://amslaurea.unibo.it/21648/}{\textit{Sviluppo di interfacce grafiche moderne in Kotlin e JavaFX: evoluzione della UI del simulatore Alchemist}}, 2020.
    \item Lorenzo Paganelli: \href{https://amslaurea.unibo.it/20540/}{\textit{Simulazione di evacuazione di folle in Alchemist: un modello di mappa mentale per pedoni cognitivi}}, 2020.
    \item Stefano Rasi: \href{https://amslaurea.unibo.it/20505/}{\textit{Manipolazione di Bytecode Java con la libreria ASM}}, 2020.
    \item Filippo Nardini: \href{https://amslaurea.unibo.it/19778/}{\textit{Sviluppo di piattaforme per il linguaggio Protelis in Kotlin e Java}}, 2019.
    \item Federico Pettinari: \href{https://amslaurea.unibo.it/19092/}{\textit{Un Framework per Simulazione e Sviluppo di Sistemi Aggregati di Smart-Camera}}, 2019.
    \item Diego Mazzieri: \href{https://amslaurea.unibo.it/19084/}{\textit{Progettazione e implementazione di agenti cognitivi per simulazioni di evacuazioni di folle in Alchemist}}, 2019.
    \item Luca Giuliani: \href{https://amslaurea.unibo.it/19071/}{\textit{Progettazione e Implementazione di un Domain Specific Language per la Costruzione di Reti Geniche}}, 2019.
    \item Manuele Pasini: \href{http://amslaurea.unibo.it/18535/}{\textit{Programmazione memory-safe senza garbage collection: il caso del linguaggio Rust}}, 2019.
    \item Nicolas Barilari: \href{http://amslaurea.unibo.it/16841/}{\textit{Programmazione Reattiva in Kotlin su sistemi Android}}, 2018.
    \item Luca Casamenti: \href{http://amslaurea.unibo.it/16788/}{\textit{Il linguaggio Ceylon}}, 2018.
    \item Davide Bondi: \href{http://amslaurea.unibo.it/15730/}{\textit{Protocollo LoRaWAN e IoT: interfacciamento con Java e sperimentazione su comunicazioni indoor}}, 2018.
    \item Matteo Magnani: \href{http://amslaurea.unibo.it/17133/}{\textit{Design e implementazione di un sistema di grid computing per il simulatore Alchemist}}, 2017.
    \item Niccolò Maltoni: \href{http://amslaurea.unibo.it/14682/}{\textit{Progettazione object-oriented di un'interfaccia grafica JavaFX per il simulatore Alchemist}}, 2017.
    \item Luca Semprini: \href{http://amslaurea.unibo.it/14673/}{\textit{Una panoramica su Kotlin: il nuovo linguaggio per lo sviluppo di applicazioni Android}}, 2017.
    \item Andrea Placuzzi: \href{http://amslaurea.unibo.it/14329/}{\textit{Integrazione dei formati di navigazione GPS standard in Alchemist}}, 2017.
    \item Giacomo Scaparrotti: \href{http://amslaurea.unibo.it/14019/}{\textit{Studio delle prestazioni del simulatore Alchemist: ottimizzazione di routing e caching}}, 2017.
    \item Elisa Casadio: \href{http://amslaurea.unibo.it/12310/}{\textit{Revisione e refactoring dell'interfaccia utente del simulatore Alchemist}}, 2016.
    \item Gianluca Grossi: \href{http://amslaurea.unibo.it/12503/}{\textit{Sviluppo di plugin per IntelliJ IDEA}}, 2016.
    \item Giovanni Romio: \href{http://amslaurea.unibo.it/10481/}{\textit{Backport di una applicazione da Java 8 a Java 7}}, 2016.
    \item Francesco Cardi: \href{http://archive.fo/zMGo8}{\textit{Sapere Adaptive Visualisation}}, 2012.
\end{innerlist}

\section{Extra-institutional teaching}
\href{https://www.bbs.unibo.eu/hp/}{\textbf{Bologna Business School}}, Bologna (BO), Italy
\begin{outerlist}
\item[] \textbf{Master Courses} %\hfill \textbf{October 2018 to December 2018}
    \begin{innerlist}
        \item \textit{Internet of Things -- Software production} --- advanced course on techniques for producing high quality software for the IoT. Focus on team coordination strategies and tools, build automation, testing, continuous integration, and continuous delivery, 4 hours.
\end{innerlist}
\halfblankline
\end{outerlist}

\href{http://www.formart.it/}{\textbf{FORMart}}, Cesena (FC), Italy
\begin{outerlist}
\item[] \textbf{Istruzione e Formazione Tecnica Superiore}
    \begin{innerlist}
        \item \textit{Internet of Things} --- Introduction to distributed computing and to the Internet of Things, with focus on Industry 4.0, 2019, 30 hours
        \item \textit{Programmazione e ICT problem solving} --- course on algorithmic problem resolution and automation, with elements of programming in Python, 2018, 30 hours
        \item \textit{Sistemi informatici e loro gestione} --- course on basics of operating systems, networking, and database management, 2017, 30 hours
        \item \textit{Elementi di Programmazione e Sviluppo di Applicazioni} --- course on imperative and object oriented programming with C and Java, 2016, 30 hours
    \end{innerlist}
\halfblankline
\end{outerlist}
