\href{http://www.unibo.it}{\textbf{Alma Mater Studiorum Università di Bologna}}, Bologna (BO), Italy
\begin{outerlist}
\item[] \textbf{PhD courses} %\hfill \textbf{June 2011 to December 2012}
    \begin{innerlist}
      \item \href{https://bitbucket.org/danysk/courses-2018-developing-maintaining-and-sharing-software-tools/downloads/}{\textit{Developing, Maintaining, and Sharing Software Tools for Research}}, \href{http://archive.fo/Fmxxm}{PhD program in Data Science and Computation}, October 2019, 8 hours.
      \item \href{http://archive.fo/HKAC6/}{\textit{DevOps for scientific research}}, \href{http://archive.fo/ppTiB}{Phd Program in Computer Science and Engineering}, September 2019, 20 hours.
      \item \href{https://bitbucket.org/danysk/courses-2018-developing-maintaining-and-sharing-software-tools/downloads/}{\textit{Developing, Maintaining, and Sharing Software Tools for Research}}, \href{http://archive.fo/Fmxxm}{PhD program in Data Science and Computation}, 2018, 20 hours.
    \end{innerlist}
\item[] \textbf{Contract professor} %\hfill \textbf{June 2011 to December 2012}
    \scriptsize{\\ \textbf{NOTE:} Students are mandatorily submitted an anonymous form where they can express their opinion about several aspects of the course in a four-valued scale (very negative, negative, positive, very positive). Such evaluations are reported here, in the form $(A, B, C, D, E)$, where:
    \\$A$: Overall course satisfaction;
    \\$B$: Availability of the teacher;
    \\$C$: Clarity of exposition;
    \\$D$: The teacher stimulates learning interest;
    \\$E$: Number of respondents.
    \\With the exception of $E$, the fraction of students evaluating positively or very positively is reported.
    }
    \normalsize
    \begin{innerlist}
      \item \href{http://archive.fo/JtEDW}{\textit{Object-Oriented Programming}}, \href{http://archive.fo/UM5wl}{Bachelor in Computer Science and Engineering}, 2019, 3 CFU/ECTS.
      \item \href{http://archive.fo/srdtN}{\textit{Object-Oriented Programming}}, \href{http://archive.fo/UM5wl}{Bachelor in Computer Science and Engineering}, 201, 3 CFU/ECTS (91.3\%, 99.0\%, 91.3\%, 91.3\%, 104).
      \item \href{http://archive.fo/54lT9}{\textit{Object-Oriented Programming}}, \href{http://archive.fo/UM5wl}{Bachelor in Computer Science and Engineering}, 2017, 3 CFU/ECTS (87.9\%, 99.1\%, 83.5\%, 94.0\%, 116).
      \item \href{http://archive.fo/0XLbd}{\textit{Object-Oriented Programming}}, \href{http://archive.fo/UM5wl}{Bachelor in Computer Science and Engineering}, 2016, 3 CFU/ECTS (82.1\%, 97.5\%, 95.1\%, 90.2\%, 123).
      \item \href{http://archive.fo/1eIMt}{\textit{Engineering Complex Adaptive Software Systems}}, \href{http://archive.fo/toz5c}{two year master in Computer Science and Engineering}, 2016, 1 CFU/ECTS (100\%, 100\%, 100\%, 100\%, 6).
      \item \href{http://archive.fo/JrWEu/}{\textit{Foundations of Informatics A}}, \href{http://archive.fo/30rN0}{Bachelor in Electronics Engineering for Energy and Information} and \href{http://archive.fo/jW52L}{Bachelor in Biomedical Engineering}, 2015, 3 CFU/ECTS (88.7\%, 95.1\%, 98.4\%, 98.4\%, 62).
      \item \href{http://archive.fo/nnsBl/}{\textit{Multi-Agent Systems}}, \href{http://archive.fo/qDVq3}{two year master in Computer Engineering}, 2011, 3 CFU/ECTS (90.9\%, 100\%, 100\%, 100\%, 11).
    \end{innerlist}
\item[] \textbf{Teaching tutor} %\hfill \textbf{June 2011 to December 2012}
    \begin{innerlist}
      \item \href{http://archive.fo/JtEDW}{\textit{Object-Oriented Programming}}, \href{http://archive.fo/UM5wl}{Bachelor in Computer Science and Engineering}, 2019.
      \item \href{http://archive.fo/srdtN}{\textit{Object-Oriented Programming}}, \href{http://archive.fo/UM5wl}{Bachelor in Computer Science and Engineering}, 2018.
      \item \href{http://archive.fo/puTDG}{\textit{Object-Oriented Programming}}, \href{http://archive.fo/UM5wl}{Bachelor in Computer Science and Engineering}, 2015.
      \item \href{http://archive.fo/5LhhW}{\textit{Engineering Complex Adaptive Software Systems}}, \href{http://archive.fo/toz5c}{two year master in Computer Science and Engineering}, 2015.
      \item \href{http://archive.fo/8jzEp}{\textit{Object-Oriented Programming}}, \href{http://archive.fo/UM5wl}{Bachelor in Computer Science and Engineering}, 2014.
      \item \href{http://archive.fo/h8JCD}{\textit{Engineering Complex Adaptive Software Systems}}, \href{http://archive.fo/toz5c}{two year master in Computer Science and Engineering}, 2014.
      \item \href{http://archive.fo/0Gr16}{\textit{Object-Oriented Programming}}, \href{http://archive.fo/UM5wl}{Bachelor in Computer Science and Engineering}, 2013.
      \item \href{http://archive.fo/XZFR0}{\textit{Foundations of Informatics A}}, \href{http://archive.fo/30rN0}{Bachelor in Electronics Engineering for Energy and Information} and \href{http://archive.fo/jW52L}{Bachelor in Biomedical Engineering}, 2013.
    \end{innerlist}
\item[] \textbf{Supervision of master theses} %\hfill \textbf{June 2011 to December 2012}
    \begin{innerlist}
      \item Giacomo Scaparrotti: \href{http://amslaurea.unibo.it/20440/}{\textit{Cross-simulator integration: ns3 as a network simulation back-end for Alchemist}}, 2020.
      \item Filippo Nicolini: \href{http://amslaurea.unibo.it/19521/}{\textit{Simulazione di Agenti BDI basati su Prolog in Alchemist}}, 2019.
      \item Matteo Francia: \href{http://amslaurea.unibo.it/13090/}{\textit{A Foundational Library for Aggregate Programming}}, 2017.
      \item Simone Costanzi: \href{http://amslaurea.unibo.it/10519/}{\textit{Integrazione di piattaforme d'esecuzione e simulazione in una toolchain Scala per aggregate programming}}, 2016.
      \item Davide Ensini: \href{http://amslaurea.unibo.it/7990/}{\textit{Spatial computing per smart devices}}, 2014.
      \item Luca Nenni: \href{http://amslaurea.unibo.it/6927/}{\textit{Simulazioni realistiche di algoritmi di Crowd Steering}}, 2014.
      \item Enrico Polverelli: \href{http://amslaurea.unibo.it/5293/}{\textit{Simulazione di algoritmi di auto-organizzazione basati su gradiente computazionale in Alchemist}}, 2012.
      \item Andrea Dallatana: \href{http://amslaurea.unibo.it/4217/}{\textit{BDI agents for Real Time Strategy games}}, 2012.
      \item Francesca Cioffi: \href{http://amslaurea.unibo.it/4088/}{\textit{Algoritmi gradient-based per la modellazione e simulazione di sistemi auto-organizzanti}}, 2012.
      \item Paolo Contessi: \href{http://amslaurea.unibo.it/4074/}{\textit{Supporting semantic web technologies in the pervasive service ecosystems middleware}}, 2012.
      \item Giacomo Pronti: \href{http://archive.fo/nBeOg}{\textit{Simulazione di ecosistemi di servizi pervasivi con supporto ad annotazioni tuple based}}, 2012.
      \item Michele Morgagni: \href{http://archive.fo/6mnSN}{\textit{Modulo di comunicazione in una infrastruttura per pervasive service ecosystems}}, 2011.
      \item Matteo Desanti: \href{http://archive.fo/rwla1}{\textit{Supporto a regole chimico-semantiche per la coordinazione di service pervasive ecosystems}}, 2011.
    \end{innerlist}
\item[] \textbf{Supervision of bachelor theses} %\hfill \textbf{June 2011 to December 2012}
    \begin{innerlist}
      \item Filippo Nardini: \href{https://amslaurea.unibo.it/19778/}{\textit{Sviluppo di piattaforme per il linguaggio Protelis in Kotlin e Java}}, 2019.
      \item Federico Pettinari: \href{https://amslaurea.unibo.it/19092/}{\textit{Un Framework per Simulazione e Sviluppo di Sistemi Aggregati di Smart-Camera}}, 2019.
      \item Diego Mazzieri: \href{https://amslaurea.unibo.it/19084/}{\textit{Progettazione e implementazione di agenti cognitivi per simulazioni di evacuazioni di folle in Alchemist}}, 2019.
      \item Luca Giuliani: \href{https://amslaurea.unibo.it/19071/}{\textit{Progettazione e Implementazione di un Domain Specific Language per la Costruzione di Reti Geniche}}, 2019.
      \item Manuele Pasini: \href{http://amslaurea.unibo.it/18535/}{\textit{Programmazione memory-safe senza garbage collection: il caso del linguaggio Rust}}, 2019.
      \item Nicolas Barilari: \href{http://amslaurea.unibo.it/16841/}{\textit{Programmazione Reattiva in Kotlin su sistemi Android}}, 2018.
      \item Luca Casamenti: \href{http://amslaurea.unibo.it/16788/}{\textit{Il linguaggio Ceylon}}, 2018.
      \item Davide Bondi: \href{http://amslaurea.unibo.it/15730/}{\textit{Protocollo LoRaWAN e IoT: interfacciamento con Java e sperimentazione su comunicazioni indoor}}, 2018.
      \item Matteo Magnani: \href{http://amslaurea.unibo.it/17133/}{\textit{Design e implementazione di un sistema di grid computing per il simulatore Alchemist}}, 2017.
      \item Niccolò Maltoni: \href{http://amslaurea.unibo.it/14682/}{\textit{Progettazione object-oriented di un'interfaccia grafica JavaFX per il simulatore Alchemist}}, 2017.
      \item Luca Semprini: \href{http://amslaurea.unibo.it/14673/}{\textit{Una panoramica su Kotlin: il nuovo linguaggio per lo sviluppo di applicazioni Android}}, 2017.
      \item Andrea Placuzzi: \href{http://amslaurea.unibo.it/14329/}{\textit{Integrazione dei formati di navigazione GPS standard in Alchemist}}, 2017.
      \item Giacomo Scaparrotti: \href{http://amslaurea.unibo.it/14019/}{\textit{Studio delle prestazioni del simulatore Alchemist: ottimizzazione di routing e caching}}, 2017.
      \item Elisa Casadio: \href{http://amslaurea.unibo.it/12310/}{\textit{Revisione e refactoring dell'interfaccia utente del simulatore Alchemist}}, 2016.
      \item Gianluca Grossi: \href{http://amslaurea.unibo.it/12503/}{\textit{Sviluppo di plugin per IntelliJ IDEA}}, 2016.
      \item Giovanni Romio: \href{http://amslaurea.unibo.it/10481/}{\textit{Backport di una applicazione da Java 8 a Java 7}}, 2016.
      \item Francesco Cardi: \href{http://archive.fo/zMGo8}{\textit{Sapere Adaptive Visualisation}}, 2012.
    \end{innerlist}
\end{outerlist}
\halfblankline

\href{https://www.bbs.unibo.eu/hp/}{\textbf{Bologna Business School}}, Bologna (BO), Italy
\begin{outerlist}
\item[] \textbf{Master Courses} %\hfill \textbf{October 2018 to December 2018}
    \begin{innerlist}
        \item \textit{Internet of Things -- Software production} --- advanced course on techniques for producing high quality software for the IoT. Focus on team coordination strategies and tools, build automation, testing, continuous integration, and continuous delivery.
\end{innerlist}
\halfblankline
\end{outerlist}

\href{http://www.formart.it/}{\textbf{FORMart}}, Cesena (FC), Italy
\begin{outerlist}
\item[] \textbf{Istruzione e Formazione Tecnica Superiore}
    \begin{innerlist}
        \item \textit{Internet of Things} --- Introduction to distributed computing and to the Internet of Things, with focus on Industry 4.0, 2019
        \item \textit{Programmazione e ICT problem solving} --- course on algorithmic problem resolution and automation, with elements of programming in Python, 2018
        \item \textit{Sistemi informatici e loro gestione} --- course on basics of operating systems, networking, and database management, 2017
        \item \textit{Elementi di Programmazione e Sviluppo di Applicazioni} --- course on imperative and object oriented programming with C and Java, 2016
    \end{innerlist}
\halfblankline
\end{outerlist}
